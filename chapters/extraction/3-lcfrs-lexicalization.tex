\documentclass[../../document.tex]{subfiles}

\begin{document}
    \section{Extracting and Lexicalizing \abrv{Lcfrs}} \label{sec:extraction:lcfrs}
    This extraction scheme for supertags with \abrv{lcfrs} rules was the product of a joint work with Richard Mörbitz \citep{MoeRup20,RupMoe21}.
    It is based on the extraction of \abrv{lcfrs} treebank grammars as shown by \citet{KalMai13} followed by a per-tree lexicalization inspired by the procedures presented by \citet{EngMalMan18}.
    This section is an extended version of the extraction described in one of our publications \citep{RupMoe21} with some adaptions to fit into the framework of this thesis.
    We describe the extraction procedure independently for each constituent tree in a treebank in five steps:\footnote{\label{footnote:lcfrs:split}
        The original formulation contained two additional steps that were called \emph{dechain} and \emph{fuseterm}.
        The first of these two removes unary non-lexical rules.
        Our formulation of binarization already merges unary nodes in the same way as this step.
        The second dismisses nullary grammar rules for the \abrv{pos} symbols and merges their terminals into parent rules.
        In contrast to the initial publication, our definition of grammar rules read from constituent structures does not involve the \abrv{pos} symbols in the first place.
        We skip both steps with equal results.
    }
    \begin{enumerate}
        \item\label{enum:lcfrs:step1}
            The constituent tree is binarized.
            The constituent symbol at each position, where the fanout of the yield is greater than one, is extended by the fanout as a subscript.
        \item\label{enum:lcfrs:step2}
            A binary \abrv{lcfrs} derivation is read from the constituent tree.
            The derivation is instantiated in the sequence of tokens of the constituent tree.
%        \item\label{enum:lcfrs:step1}
%            The derivation's leaves are dismissed and their lexical symbols are moved into their parents by each replacing a variable.
%            The \abrv{rhs} nonterminals of all rules are adjusted to accommodate the removed rules.
        \item\label{enum:lcfrs:step3}
            The derivation is transformed by \emph{transporting} the lexical symbols into unlexical rules according to a fixed strategy.
            During this step, the rules' nonterminals are annotated according to the transportation.
            Moreover, each node with rule \(r\) is replaced by a tuple \((r, s)\) where \(s\) is true if and only if the node was equipped with another lexical symbol before the step.
        \item\label{enum:lcfrs:step4}
            Now the derivation contains at most one rule equipped with two terminal symbols. Each other rule in the derivation is lexical.
            It is split by replacing one of the symbols with a variable and adding a new nullary rule.
            The result of this step is an admissible \abrv{lcfrs} supertag derivation.
        \item
            In the final step, the rules are collected as a sequence in order of the positions according to the instantiation.
            All occurring terminal symbols are replaced by a wildcard symbol ``\wildcard{}''.
            The result is a sequence of \abrv{lcfrs} supertag blueprints such that the \(i\)th blueprint is for the \(i\)th position in the sentence.
    \end{enumerate}
    After \sref{enum:lcfrs:step4}, the topmost \abrv{lhs} nonterminal is collected for each tree and set aside as an initial nonterminal for later.
    In the following, the steps (\ref{enum:lcfrs:step2}) to (\ref{enum:lcfrs:step4}) are described in more detail.
    % The section concludes with a procedure to revert the process and transform an \abrv{lcfrs} supertag derivation into a constituent structure.

    \begin{figure}
        \null\hfill
        \subfile{figures/lcfrs/ctree.tex}
        \hfill
        \begin{minipage}{.4\linewidth}
            \subcaption{\Sref{enum:lcfrs:step1} constructs a binary constituent structure with fanout annotations. This example shows a constituent structure obtained with right-branching binarization from the running example shown in \cref{fig:ex:binarization:5}.}
        \end{minipage}

        \vspace{.3cm}

        \begin{minipage}{.4\linewidth}
            \subcaption{
                \Sref{enum:lcfrs:step2} constructs a binary \abrv{lcfrs} derivation for the constituent structure above.
                Gray dashed arrows indicate the transportation of terminal symbols in the following \sref{enum:lcfrs:step3}.
            }
        \end{minipage}
        \hfill
        \subfile{figures/lcfrs/step2.tex}
        \hfill\null

        \vspace{.3cm}

        \null\hfill
        \subfile{figures/lcfrs/step3.tex}
        \hfill
        \begin{minipage}{.4\linewidth}
            \subcaption{
                Each rule has at least one terminal symbol after the transportation in \sref{enum:lcfrs:step3}.
                The nodes of the tree are not illustrated as tuples but as the rules found in the first component where the superscript ``$^{\mathrm{swap}}$'' is added if and only if the transportation marker in the second component is ``true''.
                The bottom-left-most rule in the derivation is equipped with two terminals.
            }
        \end{minipage}

        \vspace{.6cm}

        \begin{minipage}{.4\linewidth}
            \subcaption{
                After splitting the rule with two terminals in \sref{enum:lcfrs:step4}, each rule in the derivation is lexical.
            }
        \end{minipage}
        \hfill
        \resizebox{.46\linewidth}{!}{\subfile{figures/lcfrs/step4.tex}}
        \hfill\null

        \vspace{0cm}

        \null{\hfill}
        {\scriptsize\subfile{figures/lcfrs/step5.tex}}
        \hfill
        \begin{minipage}{.4\linewidth}
            \subcaption{
                The final step transforms the \abrv{lcfrs} supertag derivation into a sequence of \abrv{lcfrs} supertag blueprints.
            }
        \end{minipage}

        \vspace{.3cm}


        \caption{\label{fig:extraction:lcfrs}
            Examples for the steps (\ref{enum:lcfrs:step1}) to (\ref{enum:lcfrs:step5}) in the extraction for \abrv{lcfrs} supertags.
        }
    \end{figure}


    \paragraph{\Sref{enum:lcfrs:step2}.}
    Let \((\xi, \pi, w)\) be a binary constituent tree.
    We use the usual techniques for \abrv{lcfrs} rule induction to read a derivation \(d\) from the constituent structure \(\xi\).
    There are no rules involving the \abrv{pos} symbols \(\pi\) in the derivation \(d\).
    The terminal symbols in \(d\) are the leaves in \(\xi\). We can consider it as an implicit instantiation in the string \(w\).
    Specifically, \(d\) is a tree over \abrv{lcfrs} rules such that \(\pos(d) = \npos(\xi)\) and at each position \(\rho \in \pos(d)\) the rule \(d(\rho) = r\) is as follows:
    \begin{compactitem}
        \item The \abrv{lhs} nonterminal in \(r\) is \(\xi(\rho)\).
        \item The \abrv{rhs} nonterminals in \(r\) are \(\xi(\rho')\) for each child \(\rho'\) of \(\rho\) in \(\npos(\xi)\) and ordered from left to right.
        \item The composition in \(r\) is \(\mathrm{comp}(\yield(\xi|_\rho), \vec{ys})\) where \(\vec{ys}\) is the sequence of yields at each inner child node from left to right.
    \end{compactitem}
    By the shape of the binary constituent structure \(\xi\), each rule in \(d\) is either
    \begin{inparaitem}[]
        \item nullary and its composition contains two terminals,
        \item unary and its composition contains one terminal or
        \item binary and its composition contains no terminal.
    \end{inparaitem}


    \paragraph{\Sref{enum:lcfrs:step3}.}
    Let us consider an \abrv{lcfrs} derivation \(d\) obtained by the previous step.
    Consider an occurrence of a binary rule $A \to c(B_1, B_2)$ at position \(\rho\) in \(d\).
    This step aims to move the first terminal from the yield of the right subtree below \(\rho\) into the composition \(c\).
    We descend into the subtree and transport terminal symbols accordingly while adjusting the compositions and adding annotations to the nonterminal symbols.
    The reason for these annotations is twofold:
    \begin{itemize}
        \item In the case that the transport changes the length of a composition, the annotation is necessary to acclimate the assigned fanout for the \abrv{lhs} nonterminal.
        \item
            During the work on the extraction procedure, we had the intuition that it could be helpful to distinguish nonterminals at subtrees that were affected by the transportation and missed a terminal symbol and those that were left unchanged.
    \end{itemize}
    Moreover, a marker indicates if terminal symbols were swapped out in a unary rule during the transportation.
    It is relevant every time the first symbol in a subtree's yield coincides with a terminal in a unary rule because it must be replaced to keep it lexical.
    This information is used to undo the transportation when a constituent tree is constructed for a derivation of supertags.

    This step is described as a transformation procedure that is applied at each position \(\rho\) in \(d\) labeled by a binary rule (in any order).
    During the procedure, we keep track of a set of positions \(S\) for the aforementioned swap marker.
    Consider the occurrence of the leftmost nullary rule at position \(\underline{\rho}\) that is reachable via the second successor of \(\rho\).
%    For example, in \cref{fig:propterm:pre}, the two binary rules (\(r\)) are end points of gray arrows; these arrows start at the mentioned leaves (\(t\)).
    For each node \(\mu\) on the path from \(\underline{\rho}\) to \(\rho\) (from bottom up):
    \begin{itemize}
        \item If \(\mu\) is \(\underline{\rho}\), we remove the leftmost terminal in the rule's composition at \(\mu\).
        \item
            If \(\mu\) is neither \(\underline{\rho}\) nor \(\rho\), we insert the last removed terminal right before the variable \(\x^1_1\) and then remove the leftmost terminal in the rule's composition at \(\mu\).
            If the inserted and removed terminals are not equal, then the position \(\mu\) is added to the set \(S\).
        \item If \(\mu\) is \(\rho\), we insert the last removed terminal right before the variable \(\x^1_2\) in the rule's composition at \(\mu\).
    \end{itemize}
    Consider the case \(\mu \neq \rho\), and let \(\hat{\mu}\) be the parent of $\mu$ and $\mu$ the \(i\)th child of $\hat{\mu}$.
    If, after the removal of the first terminal in the composition at \(\mu\), the first component in the composition is empty:
    \begin{itemize}
        \item we annotate the \abrv{lhs} nonterminal at \(\mu\) and the \(i\)th \abrv{rhs} nonterminal at \(\hat{\mu}\) with $^{-1}$ and remove the empty component, and
        \item if \(i = 1\) (resp.\@ \(i = 2\)), we remove \(\x^1_1\) (resp.\@ \(\x_2^1\)) and replace every other occurrence of \(\x_1^i\) by \(\x_1^{i-1}\) (resp.\@ \(\x_2^j\) by \(\x_2^{j-1}\)) at $\hat{\mu}$.
    \end{itemize}
    Otherwise, we annotate the nonterminals with \(^{\pm 0}\).

    With the strategy to transport the first symbol in the second successor, each position can only assume the role of \(\mu\) at most once.
    Moreover, there is a suitable leaf $\underline{\rho}$ for every position $\rho$.
    As we start with two terminals in the compositions of each leaf node, the rules at \(\rho\) and \(\underline{\rho}\) become both lexical and the number of terminals in each rule between them does not change.
    To conclude the step, we replace the rule \(r\) at each position \(\rho\) with the tuple \((r, \rho \in S)\) where the second component is ``true'' if and only if \(\rho \in S\).

%    Instead of the procedure, we can also describe the result of this step in terms of properties that characterize each node.
%    Let \(M = \pos(d) \cap {\DN_+}^* \cdot \{2\} \cdot {\DN_+}^*\) denote the set of positions that are in the right subtree below any binary node and are thus affected by transportation.
%    The result for this step is a tree \(t\) over \abrv{lcfrs} supertags with \(\pos(t) = \pos(d)\).
%%    Let us denote the \abrv{lcfrs} derivation in \(t\), i.e., the projection into the first components for each node in \(t\), by \(t'\).
%    Each position \(\rho\) in \(t\) is a tuple \((r, s)\) as follows:
%    \begin{compactitem}
%        \item If \(\rho \notin M\), then the \abrv{lhs} and composition in \(r\) are equal to those in \(d(\rho)\) and \(s\) is ``false''.
%        \item
%            Otherwise, let \(\pi\) be the set of positions in \(d|_\rho\) (i.e.\@ the set of symbols in the strings \(\yield(d|_\rho)\)) and \(i = \min \pi\).
%            Then \(\pi' = \pi \setminus \{i\}\) is the set of positions in \(t|_\rho\).
%            The \abrv{lhs} nonterminal in \(r\) is that in \(d(\rho)\) plus the annotation \(^{\pm 0}\) if \(gaps(\pi') = gaps(\pi)\), otherwise plus the annotation \(^{-1}\) and it holds \(gaps(\pi') = gaps(\pi)-1\).
%            The composition in \(r\) is unambiguously determined by \(\mathrm{comp}\) via \(\pi'\) and the sets of positions in the subtrees below \(\rho\) in \(t\).
%            The Boolean \(s\) is ``true'' if and only if \(d(\rho)\) is unary and \(i\) is the lexical symbol in \(d(\rho)\).
%    \end{compactitem}


    \paragraph{\Sref{enum:lcfrs:step4}.}
    After the previous step, there is at most one leaf where the \abrv{lcfrs} composition contains two terminal symbols, either of the form \(A \to (\sigma_1 \sigma_2)\) or \(A \to (\sigma_1, \sigma_2)\).
    It is split into a chain of two, one unary and one nullary rule, such that the bottom nullary rule contains the first/left, and the top rule contains the second/right symbol.\footnote{
        In the previous formulation of the extraction procedure, it was the other way around: the top rule contained the left symbol and vice versa.
        It was swapped later to be consistent with the transportation strategy.
    }
    For instance, we obtain either the unary rule \(A \to (\x_1^1, \sigma_2)\;(\text{arg}(A))\) or \(A \to (\x_1^1 \sigma_2)\;(\text{arg}(A))\) and the nullary rule \(\text{arg}(A) \to (\sigma_1)\).
    The symbol \(\text{arg}(A)\) is a new nonterminal symbol constructed from the parent \abrv{lhs} nonterminal \(A\).
    In conclusion, each rule in the derivation is equipped with precisely one position in the token sequence.
\end{document}