\documentclass[../../document.tex]{subfiles}

\begin{document}
    \section{Guided Extraction} \label{sec:extraction:guided}
    We implemented the extraction of \abrv{lcfrs} supertags as described in the previous section and published results in parsing with them \citep{RupMoe21}.
    During the evaluation, we noted that the extracted sets of supertags were rather large and their prediction less accurate than other publications involved with supertagging.
    Following up, we published a generalization of the extraction procedure with the aim of tackling these issues \citep{Rup22}.
    Specifically, the generalization deals with the following two limitations:
    \begin{itemize}
        \item
            Constructing lexical \abrv{lcfrs} rules picks a sentence position for each inner node of the constituent structure according to a fixed strategy (cf.\@ step \ref{enum:lcfrs:step3} in \cref{sec:extraction:lcfrs}).
            Such a strategy is now formalized by a \emph{guide} that maps each inner node position of a constituent structure to sentence position in its yield.
            The concept is generalized by introducing multiple strategies to define guides for a given constituent structure called \emph{guide constructors}.
        \item
            LCFRS rules are constructed with constituent symbols as nonterminals, which are then supplemented with annotations during the lexicalization process.
            This section decouples the nonterminals from the other extraction processes and introduces multiple strategies to define them, called \emph{nonterminal constructors}.
    \end{itemize}
    This section picks up on the published generalizations and describes the extraction procedure in a more formal setting with \abrv{lcfrs}/\abrv{dcp} hybrid grammar rules.
    Similar to the previous section, this extraction process is characterized by four consecutive steps:
    \begin{enumerate}
        \item\label{extraction:hg:ranktrans}
            The tree is processed by a rank transformation that is compatible with the chosen guide constructor.
            In some cases, applying no rank transformation at all may be a viable choice.
        \item\label{extraction:hg:guide}
            A guide constructor selects a sentence position for each constituent node.
        \item\label{extraction:hg:alg}
            An admissible derivation of lexical hybrid grammar rules is constructed using the constituent tree, the guide from the previous step, and a nonterminal constructor.
            The \abrv{lhs} nonterminal at the root of the derivation is collected as an initial nonterminal.
        \item
            In the final step, the rules are collected as a sequence in the order of their lexical symbols.
            All lexical symbols are replaced by a wildcard symbol ``\tn{*}''.
            In case we aim to extract \abrv{dcp} supertag blueprints, we replace each hybrid grammar rule of the form \(A \to c_1\,c_2\,(\vec{B})\) with the \abrv{dcp} supertag \((A \to c_2 (\vec{B}), |c_1|))\) that does only contain the \abrv{dcp} composition but keeps the length of the \abrv{lcfrs} composition as fanout restriction.
    \end{enumerate}
    The rank transformations for step \ref{extraction:hg:ranktrans} are covered in \cref{sec:binarization}.
    \Cref{sec:guides} introduces the guide constructors used in step \ref{extraction:hg:guide} and explains which rank transformations are allowed in conjunction with each guide constructor.
    The concept of nonterminal constructors and some examples are discussed in \cref{sec:ntconstructors}; the remainder of \cref{extraction:hg:alg} is explained in \cref{sec:extractionalg}.
    

    \subfile{41-guides.tex}
    \subfile{42-nonterminals.tex}
    \subfile{43-extractionalg.tex}

    \ifSubfilesClassLoaded{%
        \printindex
        \bibliography{../references}%
    }{}
\end{document}