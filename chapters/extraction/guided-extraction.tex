\documentclass[../../document.tex]{subfiles}

\begin{document}
    \section{Guided Extraction} \label{sec:extraction:guided}
    In the evaluation of supertagging with lexical \abrv{lcfrs} according to the extraction in the previous section, it was already noted that the extracted sets of supertags were rather large and their prediction less accurate compared to other publications in the field. \cite{RupMoe21}
    In this section, we focus on a generalization of the extraction of lexical grammar rules as introduced by \citet{Rup22} that were investigated with the aim to tackle these two issues.
    The generalization deals with the following two limitations of the process described in the previous section:
    \begin{compactenum}
        \item
        Constructing lexical LCFRS rules picked a sentence position for each inner node of the constituent structure according to a fixed strategy.\todo{link to the subsection/paragraph}
        Such a strategy is now formalized by a \emph{guide} that maps each inner node position of a constituent structure to sentence position in its yield.
        The concept is generalized by introducing multiple strategies to define guides for a given constituent structure, which are called \emph{guide constructors}.
        \item
        LCFRS rules were constructed with constituent symbols as nonterminals, which were then supplemented with annotations during the lexicalization process.
        This section decouples the nonterminals from the other extraction processes and introduce multiple strategies to define them, called \emph{nonterminal constructors}.
    \end{compactenum}
    To implement the two generalizations, the extraction described in this section will produce \abrv{dcp} and hybrid grammars.
    \todo{describe the different formalisms: supertags as tuples with lcfrs rules <-> hybrid grammar rules}

    \subsection{Guides and Guide Constructors}
    The concept of guides for the extraction was introduced by \citet{Rup22}.
    In that publication, a guide is a function \(\varphi\) that assigns a sentence position to each inner node position \(\rho\) of a binary constituent structure \(\xi\) such that
    \begin{inparaenum}
        \item \(\varphi(\rho)\) is in the yield below the position \(\yield(\xi|_\rho)\), and
        \item \(\varphi\) is injective, i.e.\@ each sentence position is assigned to at most one inner node position.
    \end{inparaenum}
    As \(\xi\) is binary, there is exactly one sentence position that is not assigned to any inner node.
    During the grammar extraction process, that constructs a rule for each inner node position in the constituent structure, the guide is used to determine the lexical symbol in the rule.
    Intuitively, it can be seen to assign a ``responsibility'' for each inner node by a sentence position.
    The remaining sentence position is handled by a special rule that produces no constituent node.

    In this thesis, we will generalize the guide concept by relaxing the second condition: each sentence position may be mapped to multiple inner node positions in the constituent structure.
    This allows us to define rules that produce multiple constituent nodes.
    \todo[inline]{
        The concept can be generalized further: decouple the guides in to two, a lexical decomposition for the string grammar derivation and a guide that constructs the tree grammar.
        That would be able to derive hybrid grammars with, for example, strictly-inorder guided string derivations and head-driven \abrv{dcp} functions.
    }
    We will use this in \cref{sec:extraction:guided:head} to derive lexical grammars for constituent structures that are inspired by those extracted for dependency parsing.
    They are extracted such that each rule contains a lexical symbol in tandem with each inner node position of the constituent structure that it is a head of.

    \begin{definition}[Guide]
        Let \(\xi\) be an indexed tree.
        A \deflab{guide} for \(\xi\) is a function \(\varphi\colon \npos(\xi) \to \yield(\xi)\) if, for each \(\rho \in \npos\)
        \begin{inparaenum}
            \item \(\varphi(\rho) \in \yield(\xi|_\rho)\), and
            \item if there is any child position \(\rho' \in \npos(\xi)\) of \(\rho\) with \(\varphi(\rho) = \varphi(\rho')\), then each position \(\omega\) between \(\rho\) and \(\rho'\) has the same value \(\varphi(\rho) = \varphi(\omega) = \varphi(\rho')\).
        \end{inparaenum}
        \(\guide(\xi)\) denotes the set of all guides for the indexed tree \(\xi\).
    \end{definition}

    The second constraint in the definition ensures that there are no intermediate nodes assigned to a different position between an ancestor and an antecedent position in the constituent structure.
    So the guide may only assign the same leaf to connected node positions.

    \bigskip\noindent
    \begin{minipage}[t]{.7\linewidth}
        \parexample*
        Consider the constituent structure \(\xi\) illustrated in the right figure.
        The gray encircled integers shown next to each inner node give the value of a guide \(\varphi\) for the node's position.
        For example \(\varphi(\varepsilon) = \varphi(1) = \varphi(1\,1) = 4\).
        The leaves 1, 2 and 6 are not assigned to any inner node position.
    \end{minipage}
    \hfill
    \subfile{figures/guides/example-constituents.tex}


    \begin{definition}[Guide Constructor]
        Let \(\varSigma\) be an alphabet of constituent symbols.
        A \deflab<guide>[gconst]{guide constructor} is a function \(\varPhi\colon \itrees^\varSigma \parto (\DN^* \parto \DN)\) such that for each indexed tree \(\xi\) in the domain \(\dom(\varPhi)\), the partial function \(\varPhi(\xi)\) is a guide constructor for \(\xi\).
    \end{definition}

    Guide constructors are defined as partial functions, because some of them can only be sensibly defined for binary constituent structures.
    In the following, we will give the instances for guide constructors that were investigated by \citet{Rup22} plus one additional instance (\emph{headed guide constructor}).
    Each instance \(\varPhi\) will carry a name of the form \emph{\(x\) guide constructor}.
    For each \(\xi \in \dom(\varPhi)\), the guide \(\varPhi(\xi)\) will analogously be called \emph{\(x\) guide for \(\xi\)}.

    \paragraph{Vanilla Guide Constructor.}\deflab<guide>[vguide]{vanilla guide (constructor)}[]
    This guide constructor aims to formalize the assignment of lexical items to inner nodes that is achieved by the transportation in the extraction of lexical \abrv{lcfrs} rules described in the previous section.
    Similar to the prerequisites in that section, the guide is defined for binary indexed trees.
    A guide produced by this constructor maps each node position either to the leftmost leaf that is a direct successor, or (if not available) to the leftmost leaf in the yield of its right successor.
    It is formally defined as follows:
    \[
    \mathrm{van}(\xi)(\rho) = \begin{cases}
        \min L_\rho & \text{if $L_\rho \neq \emptyset$} \\
%        \mathrm{van}(\xi)(\rho \cdot 1) & \text{if $L_\rho = \emptyset$ and $|\children(\rho) \cap \pos(\xi)| = 1$} \\
        \min\;\yield(\xi_{\rho\cdot 2}) &\text{otherwise}
    \end{cases}
    \]
    Where the set \(L_\rho\) denotes the set of leaves just below a node that were not assigned to an ancestor node:
    \begin{align*}
        L_\rho &= \big\{\xi(\rho') \in \yield(\xi) \mid \rho' \in \lpos(\xi) \cap \children(\rho) \big\} \setminus \\
        &\qquad  \big\{\mathrm{van}(\xi)(\rho') \in \yield(\xi) \mid \rho' \in \ancestors(\rho)\} \big\}
    \end{align*}
    As such, the set depends on the values of \(\mathrm{van}(\xi)\) for ancestor nodes and the assignment must be determined for each node from top to bottom.

    \begin{wrapfigure}[6]{r}{.3\linewidth}
        \centering
        \subfile{figures/running-example-1.tex}
    \end{wrapfigure}
    \parexample[Vanilla Guide]
    Consider the constituent structure \(\xi\) of the running example shown to the right.
    Starting with the topmost node, the set \(L_\rho\) is empty, because \(\ancestors(\varepsilon) = \emptyset\).
    The value \(\mathrm{van}(\xi)(\varepsilon)\) is determined by \(\min\;\yield(\xi_{2}) = \min\;\{2,3\} = 2\).
    After that, the root's two children can be processed.
    For the node at position \(1\), the set \(L_{1}\) is \(\{4\} \setminus \{2\} = \{4\}\), therefore the value of \(\mathrm{van}(\xi)(1)\) is \(4\).
    In case of position \(2\), the set \(L_{2}\) is \(\{2,3\} \setminus \{2\} = \{3\}\) and \(\mathrm{van}(\xi)(1)=3\).
    As we continue with the nodes to the bottom, the second set in the term for \(L_\rho\) grows with the values that we assigned to the positions on the path from the root to \(\rho\) and may thus exclude considered leaves in the yield below \(\rho\).

    \paragraph{Strict Guide Constructor.}\deflab<guide>[sguide]{strict guide (constructor)}[]
    This is a small adjustment of the vanilla guide constructor that removes the special case where an inner node is assigned to a leaf below it.
    As the previous, this constructor is sensibly defined for binary indexed trees.
    A guide constructed by it maps each inner node to the least leaf of the second successor.
    Each inner node of rank 1 assumes the assignment of its child.
    Formally, it is defined as follows:
    \[
%    \mathrm{strict}(\xi)(\rho) = \begin{cases}
%        \mathrm{van}(\xi)(\rho \cdot 1) & \text{if $|\children(\rho) \cap \npos(\xi)| = 1$} \\
%        \min_{i \in \yield(\rho\cdot 2)} i &\text{otherwise}
%    \end{cases}
        \mathrm{strict}(\xi)(\rho) = \min \; \yield(\xi|_{\rho\cdot 2})
    \]

    \parexample[Strict Guide]
    The values assigned by the strict guide are independent of each other.
    E.g.\@ we may determine the values \(\mathrm{strict}(\xi)(2) = \min\;\yield(\xi|_{2\,2}) = \min\;\{3\} = 3\) or \(\mathrm{strict}(\xi)(1\,1) = \min\;\yield(\xi|_{1\,1\,2}) = \min\;\{5,6\}\) without computing any other values beforehand.

    \paragraph{Least and Near Guide Constructors.}\deflab<guide>[lguide]{least guide (constructor)}[]\deflab<guide>[nguide]{near guide (constructor)}[]
    These two guides were developed for binary indexed trees with the intention to
    \begin{inparaenum}
        \item assign as few nodes as possible to leaves that are not direct children (\emph{least guide constructor}), and
        \item assign many nodes to leaves that are as close as possible (\emph{near guide constructor}).
    \end{inparaenum}
    Both of those intentions are pretty similar, but the heuristic solutions that are used in these constructors' definitions are different in some cases.
    \Cref{fig:guides:diff} shows an example.
    The leaf for each inner position is determined in both cases via a breadth-first search for the first leaf that was not assigned to
    \begin{inparaenum}
        \item a descendant node, or
        \item an antecedent node, respectively.
    \end{inparaenum}
    This is formalized in the equation
    \[
    g(\xi)(\rho) = \xi\big( \min^{\unlhd} \: \big\{\rho' \in \descendants(\rho) \cap \lpos(\xi) \mid \xi(\rho') \notin L_\rho \big\} \big)
    \]
    where \(g\) is one of \(\{\mathrm{least}, \mathrm{near}\}\), and \(\unlhd\) is the total ordering of tree positions according to breadth-first search and set \(L_\rho\) keeps track of the previously assigned leaves.
    The total order \(\unlhd \subseteq \DN^* \times \DN^*\) is defined such that \(\rho \unlhd \tau\) if and only if \(|\rho| < |\tau|\) or \(|\rho| = |\tau| \land \rho <^* \tau\).
    In case of
    \begin{inparaenum}
        \item \(g = \mathrm{least}\), it excludes all assigned leaves below \(\rho\) from the breadth-first search:
        \(L_\rho = \{ g(\xi)(\rho') \in \yield(\xi) \mid \rho' \in \descendants(\rho) \cap \npos(\xi) \}\); and if
        \item \(g = \mathrm{near}\), it excludes all leaves assigned to nodes above \(\rho\):
        \(L_\rho = \{ g(\xi)(\rho') \in \yield(\xi) \mid \rho' \in \ancestors(\rho) \cap \npos(\xi) \}\).
    \end{inparaenum}
    Similarly to the vanilla guide constructor, the dependencies for the set \(\L_\rho\) determine the order in which the values are computed:
    \begin{inparaenum}
        \item in case of the least guide constructor, the values must be determined from bottom to top,
        \item in case of the near guide constructor, the values must be determined from top to bottom.
    \end{inparaenum}


    \begin{figure}
        \null\hfill
        \begin{minipage}{.35\linewidth}
            \centering
            \subfile{figures/guides/different-least-guide.tex}
            \subcaption{least guide}
        \end{minipage}
        \hfill
        \begin{minipage}{.35\linewidth}
            \centering
            \subfile{figures/guides/different-near-guide.tex}
            \subcaption{near guide}
        \end{minipage}
        \hfill\null
        \caption{\label{fig:guides:diff}
            Example for different solutions for guides by the least and near guide constructors for the same tree \(\xi\).
            Whereas the least guide assigns the distant leaf 7 to the root node, the near guide assigns 4.
            The least guide assigns five of six nodes to direct children, the near guide only three.
        }
    \end{figure}

    \needspace{3cm}
    \begin{wrapfigure}[6]{r}{.3\linewidth}
        \centering
        \subfile{figures/running-example-1.tex}
    \end{wrapfigure}
    \parexample[Least Guide]
    The assignments for the least guide are computed from bottom to top.
    In our running example that is again illustrated to the right, we may start with the node at position \(1\,1\,2\): The set \(L_{1\,1\,2}\) is empty, as there are no inner child nodes. The assigned value \(\mathrm{least}(\xi)(1\,1\,2)\) is the leftmost leaf with shortest distance to the node, i.e.\@ \(5\).
    Next, for the node at \(1\,1\), the set \(L_{1\,1}\) contains each leaf that was assigned to an inner child, i.e.\@ it is \(\{5\}\), and the assigned value is \(\mathrm{least}(\xi)(1\,1) = 1\).

    \parexample[Near Guide]
    The construction for the near guide starts from the topmost node.
    The value \(\mathrm{near}(\xi)(\varepsilon)\) is \(2\), because it is the leftmost leaf with shortest distance.
    For the child at position \(2\), the set \(L_{2} = \{2\}\) contains the leaves that were assigned to its ancestors.
    When \(\mathrm{near}(\xi)(2)\) is determined, the leaf \(2\) is excluded as potential value, which leaves us the leaf \(3\).

    \bigskip

    Whereas the previous constructors were defined solely according to the constituent structure, the remaining two take some additional linguistic information into account: heads and modifiers.
    This also requires a special treatment with respect to binarization which is applied to the constituent structures for the two following guide constructors.
    This is explained further within the following paragraphs.

    \paragraph{Head Guide Constructor.}\deflab<guide>[hguide]{head guide (constructor)}[]
    As the name suggests, these guides assign the head position to each inner node of the constituent structure.
    This implements an intuitive approach to deal with the relation between the inner nodes and their assigned head position: For each lexical symbol, there will be a lexical rule that contains the information to produce exactly the inner nodes it is head of.
    Naturally, this requires that the constituent structure is equipped with the necessary head assignment.
    And we have to keep in mind, that there is no sensible head assignment for binarized constituent trees in general.
    Therefore, we limit the this guide constructor to the cases of plain (i.e.\@ unbinarized) or head-inward ternarized (cf.\@ \cref{sec:extraction:bin:hi}) constituent trees.

    \paragraph{Dependent Guide Constructor.}\deflab<guide>[vguide]{dependent guide (constructor)}[]
    Dependent guides aim to offer leaf assignments that complement head-outward binarized constituent structures.
    They assign the head of a dependent to each inner node in the constituent structure.
    We consider these guides only for head-outward binarized constituent structures, as there is exactly one dependent for each inner node.

    \begin{figure}
        \begin{minipage}{.27\textwidth}
            \resizebox{\textwidth}{!}{\subfile{figures/guides/example-guide-van.tex}}
            \subcaption{vanilla guide}
        \end{minipage}\hfill
        \begin{minipage}{.27\textwidth}
            \resizebox{\textwidth}{!}{\subfile{figures/guides/example-guide-strict.tex}}
            \subcaption{strict guide}
        \end{minipage}\hfill
        \begin{minipage}{.27\textwidth}
            \resizebox{\textwidth}{!}{\subfile{figures/guides/example-guide-least.tex}}
            \subcaption{least guide}
        \end{minipage}

        \vspace{.5cm}

        \begin{minipage}[b]{.27\textwidth}
            \resizebox{\textwidth}{!}{\subfile{figures/guides/example-guide-near.tex}}
            \subcaption{near guide}
        \end{minipage}\hfill
        \begin{minipage}[b]{.27\textwidth}
            \resizebox{\textwidth}{!}{\subfile{figures/guides/example-guide-head.tex}}
            \subcaption{head guide}
        \end{minipage}\hfill
        \begin{minipage}[b]{.27\textwidth}
            \resizebox{\textwidth}{!}{\subfile{figures/guides/example-guide-dep.tex}}
            \subcaption{dependent guide}
        \end{minipage}

        \caption{\label{fig:guides}
            Overview over the guide constructors in an example constituent structure.
            Gray integers show the leaf assigned to each inner node for the binary constituent structure.
            Encircled leaves are not in the image of the guide.}
    \end{figure}

    \subsection{Nonterminal Constructors}
    A \emph{nonterminal constructor} computes a \abrv{lhs} nonterminal for each grammar rule.
    As each grammar rule is extracted at some inner node in the constituent structure, the constructor takes the position in the inner node into account.
    All \abrv{rhs} nonterminals are abducted from the grammar rules that are extracted in the node's children.
    The concept of nonterminal constructors resembles the nonterminal labeling strategies introduced by \citet[Sec.~6.2]{Geb21}.

    \begin{definition}[Nonterminal Constructor]
        Let \(\varGamma\) be some alphabet and \(N'\) a nonempty set.
        A \deflab[nconst]{nonterminal constructor} is a partial function \(\phi\colon \itrees_\varGamma \times \DN^* \times \power(\DN_+) \to N'\) such that \(\phi(\xi, \rho, L)\) is defined for each pair of an indexed tree \(\xi \in \itrees_\varGamma\), a position \(\rho \in \pos(\xi)\), and subset of leaves \(L \subseteq \yield(\xi|_\rho)\) below \(\rho\).
    \end{definition}

    %    Each of the following constructors computes a nonterminal for the position \(\rho\) in \(\xi\) from the constituent symbol \(\xi(\rho)\), the set of leaves \(\yield(\xi|_\rho)\) below \(\rho\) and the set of leaves \(L\) assigned by \(\guide\) to the ancestors of \(\rho\).
    %    We omit the fanout subscripts if they are \(1\).
    %    We give examples using the \emph{shortest guide}, as shown in \cref{fig:constituent:bin}, for the root position \(\varepsilon\) where \(\xi(\varepsilon) = \text{SBAR+S}\), \(\yield(\xi|_\varepsilon) = \{0,\ldots,5\}\) and \(L = \emptyset\) and the position \(1.1\) of the bottom VP node where \(\xi(1.1) = \text{VP}\), \(\yield(\xi|_{1.1}) = \{0,4,5\}\), \(L = \{0,3\}\) and \(\fanout(\yield(\xi|_{1.1}) \setminus L) = 1\).

    The image of such a constructor may assume an infinite set.
    However, we only consider the extraction from (finite) treebanks.
    Therefore the set of nonterminals occurring during the extraction is guarantied to be finite.
    During the extraction procedure, the third argument \(L\) for a constructor assumes the set of leaves that are either
    \begin{inparaenum}
        \item assigned by the guide to nodes at or below \(\rho\), or
        \item below \(\rho\) and not in the image of the guide.
    \end{inparaenum}
    It is the set of leaves that occurs in the (sub-)derivation of rules extracted at and below \(\rho\).\todo{general description of the extraction algorithm and example might be helpful}
    In the following, we will define some instances of nonterminal constructors.
    Similar to the guide constructors, each instance \(\varphi\) will carry a name of the form \emph{\(x\) nonterminal constructor}.
    The nonterminal symbols constructed by \(\varphi\) are called \emph{\(x\) nonterminals}.

    \paragraph{Vanilla Nonterminal Constructor.}\deflab<nonterminal constructor>[vnont]{vanilla nonterminal (constructor)}[]
    This resembles the nonterminals occurring in the extraction in \cref{sec:extraction:lcfrs}.
    The nonterminal \(\phi(\xi, \rho, L)\) consists of the node symbol \(\xi(\rho)\), the fanout \(\fanout(\yield(\xi|_\rho))\) as subscript, and if \(L \neq \yield(\xi|_\rho)\), then the difference in fanout \(\fanout(L) - \fanout(\yield(\xi|_\rho))\) as superscript.
    This superscript indicates the difference in fanout at \(\rho\) in the original constituent tree compared to the leaves assigned to the nodes in the subtree at \(\rho\) by the guide.\todo{equip each nt constructor with a distinct symbol and image set}

    \begin{wrapfigure}[6]{r}{0pt}
        \subfile{figures/running-example-1.tex}
    \end{wrapfigure}
    \parexample[Vanilla Nonterminals]
    Let us consider the indexed tree in our running example and the constructed dependent guide illustrated to the right.
    We focus on the lower \(\cn{vp}\) node at position \(1\,1\).
    The set of leaves \(L\) consists of the two leaves \(\{1,5\}\) in the yield below the node; the leaf \(6\) is assigned to the parent at position \(1\), and therefore not an element of \(L\).
    The nonterminal is a combination of the symbol \(\xi(1\,1) = \cn{vp}\), the fanout \(\fanout(\yield(\xi|_{1\,1})) = \fanout(\{1,5,6\}) = 2\), and \(\fanout(L) - 2 = \fanout(\{1,5\})-2=0\).
    It is denoted in the form \(\cn{vp}^0_2\).

    \paragraph{Classic Nonterminal Constructor.}\deflab<nonterminal constructor>[cnont]{classic nonterminal (constructor)}[]
    This constructor is intended to produce nonterminals that are akin to usual strategies in LCFRS extraction \cite{MaierSogaard08}.
    Therefore, it omits some of the annotations from the previous instance.
    The nonterminal consists of the symbol \(\xi(\rho)\) and the fanout \(\fanout(L)\) as subscript.
    If the constituent symbol \(\xi(\rho)\) resulted from a merging unary nodes during a binarization (see \cref{sec:binarization}), i.e.\@ it is of the form \(\nt{A$_1$+A$_2$+...+A$_k$}\), then it is replaced by the topmost symbol \(\nt{A}_1\).

    \parexample[Classic Nonterminals]
    For the classic nonterminal constructed for the root in the example indexed tree above, we determine the set \(L = \yield(\xi) = \{1,\ldots,6\}\).
    The nonterminal consists of the symbol \(\cn{sbar}\) and the fanout \(\fanout(L) = 1\) and is illustrated as \(\cn{sbar}_1\).

    \paragraph{Coarse Nonterminal Constructors.}\deflab<nonterminal constructor>[coarsenont]{coarse nonterminal (constructor)}[]
    This constructor equals the classic nonterminal constructor for the most part, but it replaces the constituent symbol \(\xi(\rho)\) either by
    \begin{inparaenum}
        \item a symbol that subsumes multiple constituent symbols given in a separate table, or
        \item its first letter.
    \end{inparaenum}
    The second case (ii) roughly approximates the first one when such a table is not available.
    The coarse nonterminal constructor implements the nonterminals in coarse-to-fine parsing \cite{Cha06,??}.
%    For our examples, the nonterminals are \(\text{S}\) and \(\text{V}\).

    \parexample[Coarse Nonterminals]
    \Cref{tab:coarse-nonterminals} shows an example for a clustering of constituent symbols in grammars extracted from the english Penn Treebank provided by \citet{Cha06}.
    Creating such a table requires domain-specific knowledge about the corpus as well as the underlying language.
    \citet{DoHoSa70} give a coarse tagset used in the english \abrv{xtag} grammar.
    \citet{SchiTeuSt99} give a standardized set of \abrv{pos} symbols in tandem with a coarse categorization used in german corpora.
    They can be used to substitute the \abrv{pos} symbols in the constituent symbols that were constructed during a binarization with a horizontal markovization window \(h>0\).
    \todo[inline]{mv example table out of example? add example symbol}


    \begin{table}
        \caption{\label{tab:coarse-nonterminals}
            Clusters of constituent symbols used in a coarse-to-fine parsing approach. \citep{Cha06} The authors extract coarse grammar rules using the cluster names instead of constituent symbols as nonterminals.
            These grammar rules are used to find candidate parses which aid in a parsing process with a fine grammar using the original constituent symbols.
        }
        \medskip
        \centering
        \begin{tabular}{ll}
            \toprule
            cluster & constituent and \abrv{pos} symbols \\
            \midrule
            \nt{S} & \cn{s}, \cn{vp}, \cn{ucp}, \cn{sq}, \cn{sbar}, \cn{sbarq}, \cn{sinv} \\
            \nt{N} & \cn{np}, \cn{nac}, \cn{nx}, \cn{lst}, \cn{x}, \cn{frag} \\
            \nt{A} & \cn{adjp}, \cn{qp}, \cn{conjp}, \cn{advp}, \cn{intj}, \cn{prn}, \cn{prt} \\
            \nt{P} & \cn{pp}, \cn{prt}, \cn{rrv}, \cn{whadjp}, \cn{whadvp}, \cn{whnp}, \cn{whpp} \\
            \bottomrule
        \end{tabular}
    \end{table}


    \subsection{Extraction Algorithm}
    The process described in this section is used to extract a sequence of supertags, one for each word in a sentence, from a given constituent tree.
    We distinguish four steps (i--iv) which are executed consecutively.
    The parameters for steps (i--iii) are described in detail in the next section.

    \paragraph{(i) Binarization.}
    We construct a binary constituent tree with the usual strategies in constituent parsing \cite{KalMai13}:
    Each unary node is merged with its child (or pos symbol, if the child is a leaf), and nodes with arity \(n>2\) are split into \(n-1\) binary nodes according to the parameters described in \cref{sec:con:para}.
    After this step, the constituent tree for a sentence \(w\) is equipped with \(|w|-1\) inner nodes.
    \Cref{fig:constituent:bin} shows a binary tree resulting from binarization of the tree in \cref{fig:constituent}.

    \paragraph{(ii) Guide.}
    In this step, we define a \emph{guide} for the binary constituent structure \(\xi\), i.e.\@ a mapping \(\guide\) between inner node positions and leaves in the constituent structure.
    In the following step, a lexical LCFRS rule will be constructed for the constituent at each inner node and the assigned leaf.
    Intuitively, the guide determines which sentence position is ``responsible'' for the constituent at each position.
    Formally, a guide for \(\xi\) is an injective function \(\guide\colon \npos(\xi) \to \yield(\xi)\) such that, for each \(\rho \in \npos(\xi)\), the assigned leaf \(\guide(\rho)\) is in \(\yield(\xi|_\rho)\).
    \Cref{fig:constituent:bin} shows a guide for our example constituent structure, assigning a leaf (illustrated in gray circles) to each inner node.
    As \(\guide\) is injective and there is one less inner node than leaves in \(\xi\), there is exactly one leaf that is not in the image of \(\guide\).
    We will investigate multiple strategies, called \emph{guide constructors}, to define guides for a given constituent tree as discussed in \cref{sec:con:para}.

    \paragraph{(iii) Lexical rule induction.}
    In this step, we will construct a lexical LCFRS rule \(r'\) and the components \(c\), \(t\) and \(p\) of the supertag for each leaf in the constituent structure.
    The nonterminals in the rule are determined by a chosen hyperparameter \(\ntcon\), called the \emph{nonterminal constructor}, in terms of the constituent symbol and the guide \(\guide\) as described in \cref{sec:con:para}.
    We give examples for (the root position in) the tree in \cref{fig:constituent:bin}, the guide values shown in pentagons (shortest guide constructor) and constituent symbols as nonterminals (classic nonterminal constructor).

    We start with the leaf \(i'\) that is not in the image of \(\guide\) and define a new nonterminal \(\text{L-}A\) using the fixed string ``L-'' and the nonterminal \(A\) produced by \(\ntcon\) for the parent of \(i'\).
    In that case \(c = t = \text{None}\), \(p = \mathit{pos}(i')\), and \(r' = \text{L-}A \to [i']\).
    In \cref{fig:constituent:bin}, the leaf \(i'=2\) is not in the image of \(\guide\), it yields \(r' = \text{L-NP} \to [2]\), and \(p = \text{NN})\).

    After that, we define the following for each position \(\rho \in \npos(\xi)\) (from bottom up) and its assigned leaf \(i = \guide(\rho)\):
    \begin{asparaitem}
        \item
        The LCFRS rule \(r'\) is assembled in the usual manner from \(i\) as lexical symbol and variables for the spans formed by the leaves in each successor except those leaves that are assigned by \(\guide\) to \(\rho\)'s ancestors.
        \(\ntcon\) produces the lhs nonterminal for \(\rho\), the rhs nonterminals are the lhs nonterminals constructed for \(\rho\)'s children.
        In our example, \(r\) is assembled from the lexical symbol \(3\), the left successor's leaves are \(\{\underline{0\vphantom{,}}_{(x_1)}, \underline{4,5} _{(x_2)}\}\) and the right one's are \(\{\underline{1,2} _{(y_1)}\}\);
        hence \(r = \text{S} \to [x_1 \, y_1 \, 3 \, x_2] (\text{VP}_2, \text{NP})\).

        \item
        \(t\) is \emph{None} if the leaf \(\guide(\rho)\) is a direct child of \(\rho\), otherwise it is the index among the children where \(\guide(\rho)\) is located.
        In our example, 3 is not a child of the root, it is in its first successor; therefore \(t=1\).

        \item
        \(c\) is \(\xi(\rho)\) if it was not introduced during binarization (i.e.\@ the symbol \(\xi(\rho)\) does not contain some markers \texttt{|<..>}), otherwise it is None.
        In our example, \(c = \text{SBAR+S}\).

        \item
        \(p\) is the pos symbol at \(\guide(\rho)\) in \(\mathit{pos}\).
        In our example \(p = \text{VBD}\).
    \end{asparaitem}

    \paragraph{(iv) Supertag extraction.}
    The tuples constructed in the previous step closely resemble supertags as described in the previous subsection.
    We pull them from the constructed tree and order them according to the sentence position included in the lexical rule.
    Lastly, the sentence position in each rule is replaced by a wildcard symbol ``\_''.

    \ifSubfilesClassLoaded{%
        \printindex
        \bibliography{../references}%
    }{}
\end{document}