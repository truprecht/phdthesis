\documentclass[../../document.tex]{subfiles}

\begin{document}
    \subsection{Lexical Derivation Extraction}\label{sec:extractionalg}
    This small subsection describes how we define a derivation of lexical hybrid grammar rules using the discussed prerequisites.
    Consider a constituent structure \(\xi\), an admissible guide \(\varphi\) for \(\xi\) and a nonterminal constructor \(\psi\).
    We define the following two functions using the guide \(\varphi\) and the constiuent strcutre \(\xi\):
    \begin{itemize}
        \item
            \(\varphi^{\text{top}}\colon \pos(\xi) \parto \yield(\xi)\) is a partial function that restricts the assignments to the topmost tree positions for each leaf:
            For each \(\rho \in \pos(\xi)\), we define \(\varphi^{\text{top}}(\rho) = \varphi(\rho)\) only if there is no \(\rho' \in \ancestors(\rho)\) with \(\varphi(\rho') = \varphi(\rho)\), and otherwise it is undefined.
        \item
            The function \(L\colon \pos(\xi) \to \power(\yield(\xi))\) maps each position \(\rho\in \pos(\xi)\) to the set of all leaves that \(\varphi^{\text{top}}\) assigns to positions at or below \(\rho\):
                % \vspace{-\baselineskip}
                \begin{align*}
                    L(\rho) &= \{ i \in \yield(\xi) \mid \exists \rho' \in \descendants_\xi(\rho)\colon \varphi^{\text{top}}(\rho') = i \} \\
                            &= \yield(\xi|_\rho) \setminus \{ \varphi(\rho') \mid \rho' \in \ancestors(\rho) \}
                \end{align*}
    \end{itemize}
    The derivation for \(\xi\), \(\varphi\) and \(\psi\) is defined from top to bottom as follows:
    \begin{itemize}
        \item We start with the root position \(\rho_0 = \varepsilon\).
        \item
            For the position \(\rho_0\), let \(s\in \DN\), and \(\rho_1, \ldots, \rho_s \in \descendants_\xi(\rho_0) \setminus \{\rho_0\}\) be the sequence of topmost parallel positions below \(\rho_0\) such that
            \begin{inparaenum}
                \item \(\varphi(\rho_i) \neq \varphi(\rho_0)\) for each \(i\in[s]\), and
                \item \(\rho_1, \ldots, \rho_s\) are ordered concerning the least leaves in \(L(\rho_1), \ldots, L(\rho_s)\).
            \end{inparaenum}
            Formally, we specify the two properties as follows:
            \begin{enumerate}
                \item For each \(j \in [s]\), \(\varphi(\parent(\rho_j)) = \varphi(\rho_0) \neq \varphi^{\text{top}}(\rho_j)\), and there is no other \(\rho_{s+1}\) which fulfills this property.
                \item Each pair \(j,k \in [s]\) with \(j<k\) satisfies \(\min L(\rho_j) < \min L(\rho_k)\).
            \end{enumerate}
        \item The rule for \(\rho_0\) is \(r_{\rho_0} = A_0 \to c_1 \, c_2\; (A_1, \ldots, A_s)\) where
        \begin{itemize}
            \item \(A_i = \psi(\xi, \rho_i, L(\rho_i))\) for each \(i \in [0,s]\),
            \item \(c_1 = \mathrm{comp}(L(\rho_0), L(\rho_1), \ldots, L(\rho_s))\), and
            \item \(c_2 = \mathrm{scomp}(\xi, (\rho_0, L(\rho_0)), (\rho_1, L(\rho_1)), \ldots, (\rho_s, L(\rho_s)))\).
        \end{itemize}
        \item The derivation for \(\rho_0\) is \(d_{\rho_0} = r_{\rho_0}\,(d_{\rho_1}, \ldots, d_{\rho_s})\) where, for each \(i\in[s]\), the identifier \(d_{\rho_i}\) denotes the (recursively constructed) derivation for \(\rho_i\).
    \end{itemize}

    The derivation \(d_\varepsilon\) contains exactly one rule per leaf in \(\yield(\xi)\), and it is an admissible hybrid grammar derivation.
    For each \(i \in \yield(\xi)\), the \abrv{dcp} composition with the lexical symbol \(i\) in \(d_\varepsilon\) contains exactly the constituent symbols at the inner node positions in \(\xi\) that \(\varphi\) assigns to \(i\).
    Each derivation \(d_{\rho_0}\) that was extracted starting at the top node position \(\rho_0\) contains one rule for each leaf in \(L(\rho_0)\) such that
    \begin{itemize}
        \item the yield of the \abrv{lcfrs} projection \(\yield(\pi_{\abrv{lcfrs}}(d_{\rho_0}))\) is the linearization of \(L(\rho_0)\), and
        \item the yield of the \abrv{dcp} projection \(\yield(\pi_{\abrv{dcp}}(d_{\rho_0}))\) is the subtree \(\xi|_{\rho_0}\) except, if \(\yield(\xi|_{\rho_0}) \neq L(\rho_0)\), then the leaf in the singleton set \(\yield(\xi|_{\rho_0}) \setminus L(\rho_0)\) is replaced by the variable symbol \(\y\).
    \end{itemize}

    \begin{wrapfigure}[8]{r}{0pt}
        \subfile{figures/guides/example-guide-head.tex}
    \end{wrapfigure}
    \parexample\label{ex:extraction:head}
    The tree to the right shows our running example constituent structure \(\xi\) with the usual illustration of values assigned by the guide \(\varphi\) next to each inner node.
    The partial function \(\varphi^{\text{top}}\) restricts the assignments by \(\varphi\) to the topmost positions; this is illustrated using encircled values.
    In our example, \(\varphi^{\text{top}}\) only assigns the position \(\varepsilon\) to the leaf \(4\).
    Both of its descendants that \(\varphi\) assigns to the leaf \(4\), the positions \(1\) and \(1\,1\), are not in the domain of \(\varphi^{\text{top}}\).
    The function \(L\) maps each position to the encircled values illustrated at descendant positions.
    For instance, the root position is mapped to the set of all leaves; the set \(L(1) = \{1,2,3,5,6\}\) for its child does not contain the leaf $4$; and the set \(L(1\,1\,1)\) for the lower \nt{vp} node is \(\{1,5,6\}\).
    Let us consider the construction for the root position \(\rho_0 = \varepsilon\), and the classic nonterminal constructor denoted by \(\psi\).
    The topmost descendant positions below \(\varepsilon\) that \(\varphi^{\text{top}}\) assigns to other leaves are \(\rho_1 = 1\,1\,1\) (the bottom \cn{vp} node) and \(\rho_2 = 1\,2\) (the \cn{np} node).
    The \abrv{lhs} \(A_0\) and \abrv{rhs} nonterminals \(A_1, A_2\) are determined from the constituent symbols at these three positions, respectively, as well as the fanouts of the three sets \(L(\rho_0) = [6]\), \(L_(\rho_1) = \{1,5,6\}\) and \(L(\rho_2) = \{2,3\}\); they are \(A_0 = \nt{sbar}_1\), \(A_1 = \nt{vp}_2\) and \(A_2 = \nt{np}_1\).
    The \abrv{lcfrs} composition is determined from three sets \(L(\rho_0)\), \(L_(\rho_1)\) and \(L(\rho_2)\) in the usual manner, it is \(c_1 = \mathrm{comp}([6], \{1,5,6\}, \{2,3\}) = (\x_1^1\x_2^1\,4\,\x_1^2)\).
    The \abrv{dcp} composition is constructed from the nodes in \(\xi\) between \(\rho_0\) and its descendants \(\rho_1\) and \(\rho_2\), it is \(c_2 = \mathrm{scomp}(\xi, (\varepsilon, [6]), (1\,1\,1, \{1,5,6\}), (1\,2, \{2,3\})) = \cn{sbar}(\cn{s}(\cn{vp}(\x_1(\varepsilon), 4), \x_2(\varepsilon)))\,\y\).
    The two successors in the derivation are determined in the same fashion starting with \(\rho_1\) and \(\rho_2\).
    In the end, we obtain the following derivation for the whole tree.
    \begin{center}
        \subfile{figures/hybrid/headed.tex}
    \end{center}
    \exampleqed

    \subsection{Viable Combinations of Extraction Parameters}\label{sec:viable-paramters}
    \Cref{sec:binarization,sec:ntconstructors,sec:guides} introduced parameters for the extraction procedure.
    As we have seen in \cref{sec:guides}, the guide constructors expect different prerequisites such that not all combinations of parameter values are compatible.
    The vanilla, strict, least, and near guide constructors make no assumptions regarding the binarization strategy and can be freely combined.
    However, we suspect the vanilla and strict guide constructors are best combined with right-branching binarization because these guides are defined via the right successor of each node.

    The dependent guide constructor expects a binary constituent structure complemented by a head assignment; thus, it can only be coupled with the head-outward binarization strategy.
    Similarly, the head guide constructor expects a head assignment and cannot be coupled with the left- or right-branching binarization strategy.
    Formally, the head-outward binarization strategy produces viable inputs for the head guide constructor, but the canonical head assignments are not helpful when it comes to the extraction:
        A head guide combines all artificial nodes introduced during the binarization and assigns them the same lexical head; hence, the effects of the rank transformation are void.
    The following example illustrates this setting; it produces a rule of rank four that combines all nodes introduced during the head-outward binarization of the former root labeled \cn{s}.
    In our experiments, we use the head guide constructor exclusively in case of no applied rank transformations or case of the head-inward transformation.

    \begin{wrapfigure}[7]{r}{0pt}
        \subfile{figures/hybrid/pilot-ho-head.tex}
    \end{wrapfigure}
    \parexample[Continues \cref{fig:ex:markovization}]
    Consider the constituent structure after applying the head-outward binarization strategy from the previous example and the values assigned by its head guide.
    All nodes constructed from the \cn{s} constituent are assigned to the lexical head \(3\).
    For the root position $\rho_0 = \varepsilon$, four descendant positions are assigned to different lexical heads: $\rho_1 = 1\,1\,1$, $\rho_2=1\,1\,2$, $\rho_3 =1\,2$ and $\rho_4=2$.
    Three of them, namely $\rho_1$, $\rho_2$ and $\rho_4$, are positions of the leaves 1, 2, and 6, respectively; the position \(\rho_3\) points to the subtree with the $\cn{np}$ node.
    The compositions constructed for the root position assume four arguments; the \abrv{dcp} composition contains all the artificial nodes introduced during the binarization.
    The following illustration shows the resulting derivation.
    \begin{center}
        \subfile{figures/hybrid/pilot-ho-head-deriv.tex}
    \end{center}
    \exampleqed

    Aside from these restrictions of compatible binarization strategies and guide constructors, the nonterminal constructors and Markovization parameters can be chosen freely.
    They determine the structure and number of the extracted supertag blueprints.
    \Cref{sec:gridsearch,sec:gridsearch:other} show a vast series of experiments that extract supertag blueprints using many combinations of extraction parameters.
\end{document}