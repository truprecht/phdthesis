\documentclass[../../document.tex]{subfiles}

\begin{document}
    \subsection{Nonterminal Constructors}\label{sec:ntconstructors}
    A \emph{nonterminal constructor} computes \abrv{lhs} and \abrv{rhs} nonterminals for each grammar rule.
    Our extraction algorithm constructs rules at inner nodes in a constituent structure; so each nonterminal constructor is supplied with the constituent structure as well as a position in it to determine a nonterminal symbol.
    The concept of nonterminal constructors resembles the nonterminal labeling strategies introduced by \citet[Section~6.2]{Geb20}.

    \begin{definition}[Nonterminal Constructor]
        Consider an alphabet of constituent symbols \(\varGamma\), a nonempty set of nonterminal symbols \(N\), and an \(\DN_+\)-indexed partition of \(N\) that groups the nonterminals by fanout \((\mathit{fo}_i \subseteq N \mid i \in \DN_+)\).
        A \deflab{nonterminal constructor} is a partial function \(\psi\colon \itrees_\varGamma \times \DN^* \times \power(\DN_+) \parto N\) such that \(\psi(\xi, \rho, L) \in \mathit{fo}_{\fanout(L)}\) is well-defined for each pair of an indexed tree \(\xi \in \itrees_\varGamma\), a position \(\rho \in \pos(\xi)\), and subset of leaves \(L \subseteq \yield(\xi|_\rho)\) below \(\rho\).
    \end{definition}

    During the extraction procedure, the third argument \(L\) for a constructor assumes the set of leaves that are either
    \begin{inparaenum}
        \item assigned by the guide to nodes at or below \(\rho\), or
        \item below \(\rho\) and not in the image of the guide.
    \end{inparaenum}
    It is the set of leaves occurring in the (sub-)derivation of rules extracted at and below \(\rho\).
    The partition \(\mathit{fo}\) and its connection to the definition of each nonterminal constructor assure that the constructed nonterminals are considered with consistent fanouts.
    In the following, we will define some instances of nonterminal constructors.
    Similar to the guide constructors, each instance will carry a name of the form \emph{\(x\) nonterminal constructor}.
    The nonterminal symbols constructed by \(\psi\) are called \emph{\(x\) nonterminals}.

    \paragraph{Vanilla Nonterminal Constructor.}
    This resembles the nonterminals in the extraction in \cref{sec:extraction:lcfrs}.
    Each nonterminal is of the form \(A_{n}^{d}\) where \(A \in \varGamma\) is a constituent symbol, \(n \in \DN_+\) is a fanout in the yield of the constituent tree, and \(d \in \{-1,0,+1\}\) is the difference between the fanout of the extracted rule and \(n\).
    The set of \emph{vanilla nonterminals} \(N\) is partitioned such that \(\mathit{fo}_{\hat{n}} = \{ A^d_{\hat{n}-d} \in N \mid A \in \varGamma, d \in [-1,1], \hat{n}-d > 0 \}\) for each \(\hat{n} \in \DN_+\).
    The \deflab{vanilla nonterminal constructor} is a function \(\psi\) such that \(\psi(\xi, \rho, L) = \xi(\rho)_{n}^{d}\) where \(n = \fanout(\yield(\xi|_\rho))\) and \(d = \fanout(L) - n\) for each indexed tree \(\xi \in \itrees_\varGamma\), position \(\rho \in \pos(\xi)\) and \(L \subseteq \yield(\xi|_\rho)\).
    As usual, we will abbreviate the nonterminal symbols by omitting the subscript if it is \(1\).

    \begin{wrapfigure}[6]{r}{0pt}
        \subfile{figures/guides/example-guide-dep.tex}
    \end{wrapfigure}
    \parexample[Vanilla Nonterminals]
    Let us consider the indexed tree in our running example and its dependent guide illustrated to the right.
    We focus on the lower \(\cn{vp}\) node at position \(1\,1\).
    The set \(L= \{1,5\}\) is the set leaves below the node minus the leaf \(6\), which is assigned to the parent at position \(1\).
    The nonterminal is a combination of the symbol \(\xi(1\,1) = \cn{vp}\), the fanout \(\fanout(\yield(\xi|_{1\,1})) = \fanout(\{1,5,6\}) = 2\), and \(\fanout(L) - 2 = \fanout(\{1,5\})-2=0\).
    It is denoted in the form \(\cn{vp}^0_2\).
    \exampleqed

    \paragraph{Classic Nonterminal Constructor.}
    This constructor intends to produce nonterminals akin to usual strategies in \abrv{lcfrs} extraction \citep{MaierSogaard08}.
    It combines the superscript and subscript of the vanilla nonterminals and annotates constituent symbols with their sum.
    Moreover, each occurrence of a constituent symbol that is the product of a merge during a rank transformation (cf.\@ merging a chain of unary nodes in \cref{sec:binarization}) is replaced by the topmost constituent symbol of such chain.
    We define a modified set of constituent symbols \(\varGamma' = \{ f(\gamma) \mid \gamma \in \varGamma \}\) where \(f(\gamma)\) is the symbol obtained by replacing each occurrence of the form \(\nt{A$_1$+A$_2$+...+A$_k$}\) in \(\gamma\) by \(\nt{A$_1$}\).
    The set of \emph{classic nonterminals} is \(N = \{ A_n \mid A \in \varGamma', n \in \DN_+ \}\) and the partition \(\mathit{fo}_n = \{A_n \mid A \in \varGamma'\}\) for each \(n \in \DN_+\).
    The \deflab{classic nonterminal constructor} is the function \(\psi\) with \(\psi(\xi, \rho, L) = f(\xi(\rho))_{\fanout(L)}\) for each indexed tree \(\xi\), position \(\rho \in \pos(\xi)\) and set of leaves \(L \subseteq \yield(\xi|_\rho)\).

    \begin{example}[Classic Nonterminals]
        Using the classic nonterminal constructor for the root in the example indexed tree above, we determine the set of leaves \(L = \yield(\xi) = \{1,\ldots,6\}\).
        The nonterminal consists of the symbol \(\cn{sbar}\) and the fanout \(\fanout(L) = 1\) and is illustrated as \(\cn{sbar}_1\).
    \end{example}

    \paragraph{Coarse Nonterminal Constructors.}
    This constructor equals the classic nonterminal constructor for the most part, but it replaces the constituent symbol \(\xi(\rho)\) either by
    \begin{inparaenum}
        \item a symbol that subsumes multiple constituent symbols given in a separate table or
        \item its first letter.
    \end{inparaenum}
    The latter case roughly approximates the first one when such a table is unavailable; later in \cref{sec:experiments}, we will refer to it as \emph{naive coarse nonterminal constructor}.
    The \deflab{coarse nonterminal constructor} mimics the nonterminals in coarse-to-fine parsing \citep{Cha06,Tei17}.

    \begin{wrapfigure}[6]{r}{0pt}
        \subfile{figures/guides/example-guide-dep.tex}
    \end{wrapfigure}
    \parexample[Coarse Nonterminals via Table]
    \Cref{tab:coarse-nonterminals} shows an example of a clustering of constituent symbols in grammars extracted from the English Penn Treebank provided by \citet{Cha06}.
    Creating such a table requires domain-specific knowledge about the corpus and the underlying language.
    \Citet{Doran99} give a coarse tagset used in the english \abrv{xtag} grammar.
    \Citet{SchiTeuSt99} give a standardized set of \abrv{pos} symbols in tandem with a coarse categorization used in German corpora.
    These categories can substitute the \abrv{pos} symbols in the constituent symbols that were constructed during a binarization with a horizontal Markovization window \(h>0\).
    Consider our running example to the right and the upper \cn{vp} node at position \(1\).
    The table contains the constituent symbol \cn{vp} in the cluster with the symbol \cn{s}.
    The nonterminals subscript is determined in the same manner as with the classic nonterminal constructor, it is \(\fanout(\{1,4,5,6\}) = 2\); hence the constructed coarse nonterminal is \(\nt{s}_2\).
    \exampleqed

    \begin{example}[Naive Coarse Nonterminals]
        Consider the same tree and node as in the above example.
        The naive coarse nonterminal constructor replaces the constituent symbol \cn{vp} with its first letter \cn{v}; hence the nonterminal symbol is \(\nt{v}_2\).
    \end{example}
    

    \begin{table}
        \caption{\label{tab:coarse-nonterminals}
            Clusters of constituent symbols are used in a coarse-to-fine parsing approach \citep{Cha06}.
            The authors extract coarse grammar rules using cluster names instead of constituent symbols as nonterminals.
            These grammar rules are used to find candidate parses, which aid in a parsing process with a finer grammar defined over the original constituent symbols.
        }
        \medskip
        \centering
        \begin{tabular}{ll}
            \toprule
            cluster & constituent and \abrv{pos} symbols \\
            \midrule
            \nt{s} & \cn{s}, \cn{vp}, \cn{ucp}, \cn{sq}, \cn{sbar}, \cn{sbarq}, \cn{sinv} \\
            \nt{n} & \cn{np}, \cn{nac}, \cn{nx}, \cn{lst}, \cn{x}, \cn{frag} \\
            \nt{a} & \cn{adjp}, \cn{qp}, \cn{conjp}, \cn{advp}, \cn{intj}, \cn{prn}, \cn{prt} \\
            \nt{p} & \cn{pp}, \cn{prt}, \cn{rrv}, \cn{whadjp}, \cn{whadvp}, \cn{whnp}, \cn{whpp} \\
            \bottomrule
        \end{tabular}
    \end{table}

    \paragraph{Nonterminal Constructors for \abrv{Dcp} Supertags.}
    Since \abrv{dcp} supertags do not contain \abrv{lcfrs} compositions, their nonterminals do not need to be consistent with any fanout.
    We use two additional \emph{extensions} for the classic and coarse nonterminal constructor in the case of dcp supertags:
    \begin{itemize}
        \item The \deflab{nof extension} completely removes the fanout subscripts in all constructed nonterminal symbols.
        \item The \deflab{disc extension} removes the fanout subscript if it is 1 and replaces it by the marker ``$_{\text{disc}}$'' if it is greater than 1; hence, it distinguishes nonterminals for continuous and discontinuous derivations.
    \end{itemize}
    In \cref{sec:experiments}, we refer to the extended nonterminal constructors by adding the extension name as a suffix, for instance, \emph{coarse-disc} for the coarse nonterminal constructor with the disc extension.
\end{document}