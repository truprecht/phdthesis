\documentclass[../../../../document.tex]{subfiles}

\begin{document}
    \begin{tikzpicture}[
        edge from parent path={(\tikzparentnode.south) -- ++(0,-.5ex) -| (\tikzchildnode.north)},
        every path/.style={line width=.2ex},
        level distance=4ex,
        anchor=center,
        every node/.style={font=\small}
        ]
        \matrix (pos) [matrix of nodes, every node/.style={anchor=center, inner sep=0pt}, column sep=4ex, row sep=1mm] {
            \cn{vbz} & \cn{rb} & \cn{rb} & \cn{jj} & \cn{in} & \cn{dt} \\
            \tn{is}\strut & \tn{n't}\strut & \tn{really}\strut & \tn{silent}\strut & \tn{at}\strut & \tn{all}\strut \\
        };

        \node[above=15mm of pos-1-3] {\cn{vp}}
            child{node[above=1mm of pos-1-1] {1}}
            child{node[above=1mm of pos-1-2] {2}}
            child{node (midnode) {\cn{advp}}
                child {node[above=1mm of pos-1-3] {3}}}
            child {node at (midnode-|pos-1-4) {\cn{adjp}}
                child {node[above=1mm of pos-1-4] {4}}}
            child {node {\cn{advp}}
                child {node[above=1mm of pos-1-5] {5}}
                child {node[above=1mm of pos-1-6] {6}}};

        % redraw corssing edges with white foundation
%        \begin{scope}
%            \draw[white, line width=1ex] (vp2.south) ++(-.5ex,-.5ex) -| (term4.north);
%            \draw (vp2.south) -- ++(0,-.5ex) -| (npr.north);
%            \draw (vp2.south) -- ++(0,-.5ex) -| (np1.north);
%            \draw[double] (vp2.south) -- ++(0,-.5ex) -| (term4.north);
%
%            \draw[white, line width=1ex] (vp1.south) -- ++(0,-.5ex) -| (term3.north);
%            \draw (vp1.south) -- ++(0,-.5ex) -| (vp2.north);
%            \draw[double] (vp1.south) -- ++(0,-.5ex) -| (term3.north);
%        \end{scope}
    \end{tikzpicture}
\end{document}
