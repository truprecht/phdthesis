\documentclass[../../document.tex]{subfiles}

\begin{document}
    \section{Rank Transformations}\label{sec:ranktransformations}
    This section focuses on transformations for constituent trees that restrict occurring node ranks.
    For instance, \emph{binarization} is a standard preprocessing step in constituent parsing that produces trees exclusively equipped with binary nodes.
    In the literature, it is dominantly considered as a step transforming a grammar instance \citep[e.g.\@][]{KleMan03,Son08,Kal10}, we follow \citet{CraSchBod16} and regard these transformations for constituent trees.
    We call each transformation for constituent trees that produces constituent structures with restricted ranks a \deflab{rank transformation}; binarization is an instance of a rank transformation.
    As usual, all referenced or described transformations are reversible.

    Binarization is a usual tool to produce grammar instances in normal forms, e.g., the Chomsky normal form for parsing with context-free grammar instances.
    The process was generalized to other grammar formalisms such as \abrv{lcfrs} \citep{Kal10, Cra12}.
    During the process, constituent tree nodes with a rank greater than two are broken down by introducing binary \emph{artificial nodes}.
    These artificial nodes have annotations that clearly distinguish them from the original nodes.
    There are several approaches to constructing the artificial nodes; that is, the direction in which the nodes are split.
    \Citet{Son08} investigated their influence on parsing with binary context-free grammars obtained using various such approaches.
    \Citet{KleMan03} extend binarization by \emph{Markovization} for the artificial node symbols.
    They show that the extension can produce more compact grammar instances or increase the parsing accuracy.

    In this section, we will first briefly bring up and compare well-established approaches to binarize constituent structures in \cref{sec:binarization}.
    We coarsely follow \citet{Cra12} and characterize the approaches in terms of a \deflab{rank transformation strategy} that defines the structural transformation and extensions (\emph{Markovization, directive indicators} and \emph{trailing nodes}) that alter the artificial constituent symbols introduced during this step.
    Lastly, we introduce a new strategy to construct trees with at most three successors per node, called head-inward transformation in \cref{sec:extraction:bin:hi}.

    \subsection{Binarization}\label{sec:binarization}
    The \deflab{binarization} of constituent structures removes and replaces nodes that have either one or more than two successors.
    These two cases are handled separately.

    All the binarization strategies described in this section share the approach for dealing with chains of nodes having one successor:
    \begin{inparaenum}
        \item If the successor of the bottom-most node in the chain is an inner node, then they are merged into this successor.
        \item If the successor of the bottom-most node in the chain is a leaf, they are merged with the \abrv{pos} symbol for the leaf.
    \end{inparaenum}
    The merged symbols must not occur in the original alphabets of constituent and \abrv{pos} symbols and must be unique for each sequence of merged constituent/\abrv{pos} symbols.
    Here, the merged nodes are denoted by concatenating the symbols interleaved by ``\cn{+}''.
    For instance, a chain of unary nodes ``\cn{root}'', ``\cn{s}'' that is merged with the \abrv{pos} symbol ``\cn{\$.}'' is the new symbol ``\cn{root+s+\$.}''.
    This approach of handling nodes with one successor is done after dealing with nodes with more than two successors.
    Because this is common among all binarization strategies, we view only the other case as part of the strategies.

    Each node carrying \( n > 2 \) successors is split into \( n-1 \) binary nodes that are connected in some parent-child constellation depending on the strategy.
    The topmost node of these new binary nodes carries the label of the original node.
    Each following node contains a new symbol of the form \(A\binsym{B_1, \ldots, B_k}\) where \(A\) original node's symbol and \(B_1, \ldots, B_k\) are some successor symbols determined by the binarization strategy.
    Because all binarization strategies are defined per node, we describe each binarization strategy as a function that replaces the root node in a constituent tree (if it has more than two successors) with a collection of binary nodes.
    The function is then called recursively to handle each subtree.
    The reversal of the binarization procedure is independent of the binarization strategy:
    \begin{compactenum}
        \item Each binary node with a symbol introduced during the binarization procedure is removed, and its children are appended to its parents.
        \item Each node originating from a merge of unary nodes is expanded by replacing it with the chain of unary nodes.
        \item Each \abrv{pos} symbol originating from a merge is replaced by the trailing pos symbol, and the sequence of unary nodes is inserted atop the corresponding leaf in the constituent structure.
    \end{compactenum}

    \paragraph{Left- or Right-Branching Binarization.}
    These strategies replace each node \(\rho_0\) that has \( n>2 \) successors \(\rho_1, \ldots, \rho_n\) with a series of \( n-1 \) binary nodes \(\ceta_1, \ldots, \ceta_{n-1}\).
    The \deflab{right-branching binarization}[right-branching strategy] \defabrv{right-branching binarization}{\abrv{rb}} constructs this sequence of nodes as follows:
    \begin{itemize}
        \item The first node \(\ceta_1\) has the same symbol as the original node \(\rho_0\).
        \item For each \(2 \leq i \leq n-1\), the symbol for the node \(\ceta_i\) consists of the symbol of the original node \(\rho_0\) and the symbols of its rightmost \(n-i+1\) successors \(\rho_{i}, \ldots, \rho_n\). Each of these symbols is denoted in the form \(A\binsym{B_i,B_{i+1},\ldots,B_n}\) where \(A\) is symbol at \(\rho_0\) and \(B_1, \ldots, B_n\) are the symbols at its children \(\rho_1, \ldots, \rho_n\).
        \item For each \(1 \leq i \leq n-2\), the left successor of the node \(\ceta_i\) is the node \(\rho_i\), and its right successor is \(\ceta_{i+1}\).
        \item The successors of the last node \(\ceta_{n-1}\) are \(\rho_{n-1}\) and \(\rho_{n-2}\).
    \end{itemize}
    The \deflab{left-branching binarization}[left-branching strategy] \defabrv{left-branching binarization}{\abrv{lb}} mirrors the right-branching strategy: It constructs a series of binary nodes connected via their left successors in the opposite direction.
    If a binarized constituent tree was complemented by a lexical head assignment, then it is lost after applying the left- or right-branching binarization strategy; there is no reasonable lexical head assignment that can be obtained after such binarization.
    \Cref{fig:ex:pilot} shows an example with a constituent tree that underlines the binarization strategies due to its flat structure; \cref{fig:ex:binarization} shows the binarization strategies with our running examples.

    \paragraph{Head-Outward Binarization.}
    The \deflab{head-outward binarization}[head-outward strategy] \defabrv{head-outward binarization}{\abrv{ho}} mixes the left- and right-branching strategies at each higher-ranked node depending on the successor that contains its lexical head.
    If the node is of rank \(k\) and its lexical head is at or below its \(n\)th successor, then
    \begin{enumerate}
        \item the trailing \(k-n\) successors are split according to the left-branching binarization strategy forming \(k-n-1\) artificial nodes, and
        \item the first \(n-1\) successors are split according to the right-branching binarization strategy, forming another \(n-1\) artificial nodes.
    \end{enumerate}
    In the resulting constituent tree, the lexical head is in the yield of the bottom-most constructed node.
    In bottom-up parsing processes, the bottom-most nodes are explored first, and the constituent symbol is determined.
    This binarization strategy can be advantageous in such settings, as the first explored node contains the lexical head.
    We use distinct symbols for the introduced top left-branching nodes and the bottom right-branching nodes to avoid ambiguities during parsing with the extracted grammar instances.
    In these constituent symbols, the left symbol ``\scalebox{.4}[.8]{\textless{}}'' is doubled in left-branching nodes, and the right symbol ``\scalebox{.4}[.8]{\textgreater{}}'' is doubled in right-branching nodes.
    We call these \emph{directive indicators} for artificial nodes; \cref{fig:ex:pilot,fig:ex:markovization} show examples involving these indicators.
    For each constituent tree that is binarized using the head-outward strategy, there is a canonical lexical head assignment that is obtained by additionally assigning each artificial node to its origin's head; the first and last tree in \cref{fig:ex:pilot} illustrate an example.
    The head-outward strategy is the only investigated rank transformation that obtains binary trees complemented with a lexical head assignment.

    \begin{figure}
        \null\hfill
        \subfile{figures/binarization/pilot.tex}
        \hspace{.5cm}
        \begin{minipage}{.4\linewidth}
            \subcaption{Plain constituent tree with illustrated lexical head assignment}
        \end{minipage}

        \vspace{.5cm}

        \begin{minipage}{.4\linewidth}
            \subcaption{Right-branching binarized constituent tree}
        \end{minipage}
        \hspace{.5cm}
        \subfile{figures/binarization/pilot-binary.tex}
        \hfill\null

        \vspace{.5cm}

        \null\hfill
        \subfile{figures/binarization/pilot-binary-lf.tex}
        \hspace{.5cm}
        \begin{minipage}{.4\linewidth}
            \subcaption{Left-branching binarized constituent tree}
        \end{minipage}

        \vspace{.5cm}

        \begin{minipage}{.4\linewidth}
            \subcaption{Head-outward binarized constituent tree with illustrated lexical head assignment}
        \end{minipage}
        \hspace{.5cm}
        \subfile{figures/binarization/pilot-binary-ho.tex}
        \hfill\null

        \vspace{.5cm}

        \caption{\label{fig:ex:pilot}
            A constituent tree was binarized with left-branching, right-branching, and head-outward binarization strategies.
        }
    \end{figure}

    \paragraph{Minimizing Fanout or Parsing Complexity.}
    Two binarization strategies specifically consider the properties of \abrv{lcfrs} instances extracted from constituent treebanks after binarization.
    \Citet{gomez2009optimal} criticize that the usual strategies can produce constituent nodes with larger fanout than in the original treebank.
    The parsing complexity with \abrv{lcfrs} instances depends on the fanouts, so binarization strategies may have an influence on it.
    They propose a binarization strategy that minimizes the fanouts in its results.
    \Citet{Gil10} argues that the parsing complexity not only depends on the single occurring fanouts but on the combination of fanouts in each node and its children.
    \Citeauthor{Gil10} adapts the strategy to minimize the sum of fanouts in such node triplets.
    In contrast to the previous strategies, these two scramble the order of the successors in the constructed subtrees.
    They are rarely considered in parsing, and experiments conducted by \citet{Cra12} show underwhelming results regarding parse time and accuracy.
    We do not investigate these two strategies in our experiments.

    \begin{figure}
        \begin{minipage}[b]{.3\linewidth}
            \resizebox{\linewidth}{!}{\subfile{figures/binarization/survey.tex}}
            \subcaption{\label{fig:ex:binarization:1}no transformation}
        \end{minipage}
        \hfill
        \begin{minipage}[b]{.32\linewidth}
            \resizebox{\linewidth}{!}{\subfile{figures/binarization/survey-binary.tex}}
            \subcaption{\label{fig:ex:binarization:2}right-branching binarization}
        \end{minipage}
        \hfill
        \begin{minipage}[b]{.3\linewidth}
            \resizebox{\linewidth}{!}{\subfile{figures/binarization/survey-binary-lf.tex}}
            \subcaption{\label{fig:ex:binarization:3}left-branching binarization}
        \end{minipage}

        \vspace{.5cm}
        \begin{minipage}[b]{.3\linewidth}
            \resizebox{\linewidth}{!}{\subfile{figures/binarization/hearing.tex}}
            \subcaption{\label{fig:ex:binarization:4}no transformation}
        \end{minipage}
        \hfill
        \begin{minipage}[b]{.32\linewidth}
            \resizebox{\linewidth}{!}{\subfile{figures/binarization/hearing-binary.tex}}
            \subcaption{right-branching binarization}
        \end{minipage}
        \hfill
        \begin{minipage}[b]{.3\linewidth}
            \resizebox{\linewidth}{!}{\subfile{figures/binarization/hearing-binary-lf.tex}}
            \subcaption{\label{fig:ex:binarization:6}left-branching binarization}
        \end{minipage}

        \caption{\label{fig:ex:binarization}
            Comparing the two left- and right-branching binarization strategies in our two running examples:
            \Crefrange{fig:ex:binarization:1}{fig:ex:binarization:3} show constituent structures and \abrv{pos} symbols for ``where the survey was carried out'', \crefrange{fig:ex:binarization:4}{fig:ex:binarization:6} for ``A hearing was scheduled on the issue today''.
            In both cases, head-outward binarization coincides with right-branching binarization.
        }
    \end{figure}


    \subsection{Markovization}
    The symbols introduced by all binarization strategies contain the symbol of the substituted node \(A\) as well as the symbols \(B_1, \ldots, B_k\) in from the successors.
    They are chosen to represent an intermediate state in parsing the original non-binary rule.

    \deflab{Markovization} is an extension to the binarization strategies that change the constructed labels such that they contain only the symbols of the last \(h\) successors that were recently processed.
    The value \(h\) is called the \deflab<Markovization>{horizontal Markovization context} and is considered a parameter for binarization.
    With large values for \(h\), the symbols are unchanged in the formulation of the binarization strategy.
    If we choose a small value for \(h\), this generalizes the symbols such that they are common in binarized constituent trees where nodes with the same symbol and similar successors are binarized.
    In the edge case of \(h=0\), there is no information about the children, and all symbols constructed during the binarization of a node are the same.

    Additionally to the horizontal window \(h\), there is a \deflab<Markovization>{vertival Markovization context}[vertical context] \(v\) that may add information from the parent symbols for each node in the constituent tree (and not just the nodes introduced during the binarization).
    With values greater than 1, the last \(v-1\) parents are added to each node in the constituent structure.
    A sequence of additional parent symbols is denoted by concatenating them, interleaved by ``$^\wedge$'', as a superscript to the symbol's right.
    \Cref{fig:ex:markovization} shows an example where the node carrying the symbol \cn{np} is extended by its former parent symbol \cn{s}.

    \begin{figure}
        \centering
        \subfile{figures/binarization/pilot-binary-ho-v2-h0.tex}
        \caption{\label{fig:ex:markovization}
            A binary constituent structure was obtained using the head-outward binarization strategy.
            The artificial nodes do not include information about the successor symbols. The horizontal Markovization window is \(h = 0\).
            The node labeled by \cn{np}$^{\wedge\cn{s}}$ was extended by its parent in the original constituent structure. The vertical Markovization window is \(v = 2\).}
    \end{figure}


    \subsection{Head-Inward Transformation} \label{sec:extraction:bin:hi}
    A later section deals with a grammar extraction strategy borrowed from dependency parsing, cf.\@ \emph{head guides} in \cref{sec:guides}.
    In early experiments, the grammars obtained by this strategy tended to be rather large, and the binarization strategies mentioned above either failed to decrease their size or there was no intuitive strategy to obtain lexical head assignments for the constituent trees after binarization.
    This issue is described in more detail at the end of \cref{sec:viable-paramters}.
    We came up with the \deflab{head-inward transformation}[head-inward transformation strategy] and a complementary lexical head assignment for its result that tackles this problem but does not yield binary trees in general.
    It can be seen as the opposite of head-outward binarization because when a node is expanded with artificial children, it does keep the child carrying the head at the topmost position.
    All other children are distributed as or into (artificial) child nodes to the left or right of this head-carrying child.
    The process coincides with the node splitting in right-branching binarization if the lexical head assignment strictly maps each inner node to the leftmost leaf (and vice versa).
    However, we omit to remove unary nodes in this transformation, so the results may appear different even in said cases.

    More formally:  
        Consider an inner node \(\rho\) with \(n \geq 2\) successors and whose head is located in the \(i\)th child (for some \(i \in [n]\)).
        The strategy introduces artificial binary nodes for the successors \(1, \ldots, i-1\) and \(i+1, \ldots, n\) as usual with the right-branching strategy.
    These artificial nodes and the \(i\)th child are attached to \(\rho\), composing a node with at most three non-leaf successors.
    For the resulting tree, there is a canonical lexical head assignment that preserves the head for each non-artificial node; the head of each artificial node is the head of its left successor.
    The process is extended with Markovization and directive indicators for the artificial nodes in the same manner as the binarization strategies.
    Moreover, we add an artificial (unary) node concluding each sequence of right-branching artificial nodes as a parent of the \(i-1\)th and \(n\)th successor.
    \Cref{fig:ex:head-inward} shows an example of a constituent structure obtained by this rank transformation.

    \begin{figure}
        \centering
        \subfile{figures/binarization/pilot-head-inward.tex}
        \caption{\label{fig:ex:head-inward}
            A constituent tree was obtained using the head-inward rank transformation.
        }
    \end{figure}


    \ifSubfilesClassLoaded{%
        \printindex
        \bibliography{../references}%
    }{}
\end{document}