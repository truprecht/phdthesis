\documentclass[../../document.tex]{subfiles}

\begin{document}
    \section{Supertags and Supertag Blueprints}\label{sec:supertags}\deflab<supertag>{supertag blueprint}[]
    We consider the following three forms of supertag blueprints depending on the underlying grammar formalism:
    \begin{itemize}
        \item
            Each \deflab<supertag>{\abrv{lcfrs} supertag}[\abrv{lcfrs} supertag blueprint] is a tuple \((r, b)\) where \(r\) is a nullary, unary or binary lexical \abrv{lcfrs} rule containing the terminal \wildcard{} and \(b \in \{\text{True}, \text{False}\}\) is a boolean marker.
            The marker \(b\) stores information from the lexicalization process and is needed to reverse the process to transform a derivation into a constituent tree; it is only \(\text{True}\), if \(r\) is a unary rule.
        \item
            Each \deflab<supertag>{\abrv{dcp} supertag}[\abrv{dcp} supertag blueprint] is a tuple \((r, f)\) where \(r\) is a lexical \abrv{dcp} rule containing the terminal \wildcard{} and \(f \in \DN_+\) is a positive integer.
            During the parsing process, \(f\) acts as a restriction for the fanout of all derivations rooted by a supertag instantiated for \((r,f)\).
        \item Each \deflab<supertag>{hybrid grammar supertag}[hybrid grammar supertag blueprint] is a lexical hybrid grammar rule containing the terminal \wildcard{}.
    \end{itemize}
    For each of the three underlying grammar formalisms \(F\), an \deflab{supertag}[\(F\) supertag] is of the same form as an \(F\) supertag blueprint except the terminal \wildcard{} in the grammar rule is replaced by a positive integer.
    We call the supertag that is obtained by replacing \wildcard{} in a supertag blueprint \(r\) with the integer \(i\) an \deflab<supertag>{instance of a supertag blueprint}[instance of \(r\)] and denote it by \(r[i]\).
    In the following, we consider derivations with \abrv{lcfrs} and \abrv{dcp} supertags.

    \subsection*{\abrv{Lcfrs} Supertag Derivations}
    \deflab<derivation>{\abrv{lcfrs} supertag derivation}[]
    The concept behind derivations with \abrv{lcfrs} supertags is a trivial adaption of \abrv{lcfrs} derivations, because the second component is solely used to reverse the lexicalization process that was used for the extraction.
    Consider a set of \abrv{lcfrs} supertags \(S\).
    We consider the \emph{set of derivations in \(S\)}, denoted by \(\derivs^S\), that contains the tree \(d \in \T_S\) if and only if the \abrv{lcfrs} rules within \(d\) (i.e., the first components) form a derivation.
    Analogously, the derivation \(d\in \derivs^S\) is called \deflab<admissible>{admissible \abrv{lcfrs} supertag derivation}[admissible], if the derivation of \abrv{lcfrs} rules within \(d\) is an admissible derivation; the identifier \(\aderivs^S\) denotes the set of all admissible derivations in \(S\).
    For each derivation \(d \in \aderivs^S\), we obtain the \emph{constituent tree for \(d\)} by reversing the extraction process described in \cref{sec:extraction:lcfrs} (by applying the steps \ref{enum:lcfrs:step4}--\ref{enum:lcfrs:step1} in reverse).
    We omit a detailed discussion, the process is straight-forward.

    \subsection*{\abrv{Dcp} Supertag Derivations}
    \deflab<derivation>{\abrv{dcp} supertag derivation}[]
    \abrv{Dcp} compositions do not impose any restrictions regarding admissible derivations; as opposed to \abrv{lcfrs}, which precisely specify which parts of the yield of successor derivations may occur next to each other.
    Our early implementations of parsers for \abrv{dcp} supertags struggled with the sheer amount of derivations that can be found in contrast to \abrv{lcfrs}.
    Therefore, we add restrictions for the case of parsing with \abrv{dcp} supertags by extending each supertag with a positive integer \(f\in \DN_+\) as described above.
    For each \abrv{dcp} supertag or \abrv{dcp} supertag blueprint of the form \((r,f)\), the component \(f\) is called the \deflab{fanout restriction}.
    Consider a set of \abrv{dcp} supertags \(S\).
    Analogously to \abrv{lcfrs} supertag derivations, the \emph{set of derivations in \(S\)}, denoted by \(\derivs^S\), contains each tree \(d \in \T_S\) if and only if the \abrv{dcp} rules within  \(d\) form a \abrv{dcp} rule derivation.
    For each such derivation \(d \in \derivs^S\), let \(\mathrm{lex}(d)\) denote the set of all lexical symbols within \(d\).
    We call a \abrv{dcp} supertag derivation \(d \in \derivs^S\) of the form \((r, f)\big(d_1, \ldots, d_k\big)\) \deflab<admissible>{admissible \abrv{dcp} supertag derivation}[admissible] if all its successors \(d_1, \ldots, d_k\) are admissible and the following three properties are satisfied:
    \begin{enumerate}
        \item The fanout of the set \(\mathrm{lex}(d)\) is exactly \(f\).
        \item All the following sets of lexical symbols are mutually exclusive: \(\mathrm{lex}(d_1), \ldots, \mathrm{lex}(d_k)\) and \(\mathrm{lex}(r)\). Hence, each lexical symbol in \(\mathrm{lex}(d)\) occurs exactly once within \(d\).
        \item\label{prop:dcp:admissible:order}
            The child derivations \(d_1, \ldots, d_k\) are ordered such that \(\min \mathrm{lex}(d_1) < \min \mathrm{lex}(d_2) < \ldots < \min \mathrm{lex}(d_k)\). 
    \end{enumerate}
    The identifier \(\aderivs^S\) denotes the \emph{set of admissible derivations in \(S\)}.
    For each set of derivations \(D\), we denote the set of derivations rooted in a supertag with the \abrv{lhs} nonterminal \(A\) by \(D_A\), and the set of derivations with the lexical symbols \(\pi\subseteq\DN_+\) as \(D(\pi) = \{d \in D \mid \mathrm{lex}(d) = \pi\}\).
    For each \abrv{dcp} supertag derivation \(d\), we define the \emph{constituent structure for \(d\)} as the constituent structure for the derivation of \abrv{dcp} rules (i.e., the first components) within \(d\).
\end{document}

