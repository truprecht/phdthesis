\documentclass[../document.tex]{subfiles}

\begin{document}
    \chapter{Grammar Formalisms}
    This chapter describes three grammar formalisms that we use for parsing natural language sentences.
    For the subject of this thesis, let us consider a grammar as a collection of rules that are capable of deriving either strings or trees.
    All considered grammar formalisms share the same fundamental structure of their rules:
        Each consists of a sequence of nonterminal symbols, one of which is called the left-hand side (\abrv{lhs}) nonterminal and the rest the right-hand side (\abrv{rhs}) nonterminals, and an expression that computes strings or trees based on an argument for each \abrv{rhs} nonterminal.
        The symbols, that are occurring in these expressions and are part of the result, are called terminal symbols.
    The nonterminals already hint at the recursive structure of grammar derivations: the arguments used to compute the result are computed by successor rules whose \abrv{lhs} nonterminal match the right-hand side nonterminals.

%    However, we will not use grammars to derive any strings, but the opposite:
%        We analyze strings in consideration which collection of grammar rules may be used to derive a given string.
%    This process is called parsing.
%    Usually, the result of this process is not a collection of grammar rules but a digest of it, called derivation tree.

    First, we focus on linear context-free rewriting systems (\abrv{lcfrs}), an extension of context-free grammars introduced by \citet{VijWeiJos87} and extensively studied by \citet{SekMatFujKas91}.
    In this formalism, each rule produces a tuple of strings which may occur in non-consecutive parts in a finally derived string.
    This is used to model sentence positions in the yield of discontinuous constituents.
    When \abrv{lcfrs} are used for parsing of constituency structures, we typically find a rule derivation that computes the input sentence and assume the nonterminals occurring in the derivation as constituency structure.
    \todo{Verwandschaft mit pmcfg, mcfg, tree adjoining grammars, head grammars, minimalist grammars, range concatenation grammars}
    \todo{auch notation range concatenation grammars}

    Secondly, we have a look at a subset of definite clause programs (\abrv{dcp}), which were introduced by \cite{Der85} as a logic characterization of attribute grammars.
    They were used in the context of parsing by \cite{Ned14,Geb17,Geb20} in combination with \abrv{lcfrs} forming hybrid grammars (see next paragraph).
    In contrast to context-free grammars and \abrv{lcfrs}, each rule in this grammar formalism computes a collection of trees instead of strings.
    However, in the scope of this thesis, we consider only \abrv{dcp} rules that produce consecutive parts in trees.
    During parsing with these \abrv{dcp}, we find a derivation with the rules in a grammar that produces a tree with the input sentence in its leaves and assume the tree as constituency structure.

    The last formalism among the considered are \abrv{lcfrs}/\abrv{dcp} hybrid grammars as introduced by \citet{Ned14}.
    They combine the two aforementioned grammar formalisms:
        Besides the nonterminals, each rule contains the expression of an \abrv{lcfrs} as well as the expression of a \abrv{dcp}.
    During parsing, we find a derivation such that the \abrv{lcfrs} expressions produce the input sentence, the \abrv{dcp} expressions yield the constituency structure.

    In the following sections for each of the three formalisms, the grammars are described gradually:
    \begin{inparaenum}
        \item Each section starts with the expressions contained in each rule that compose strings or trees.
        \item After that, they express the rules and the other components in a grammar.
        \item Finally, they explain the derivations in a grammar in tandem with their computed objects.
    \end{inparaenum}
    Each of these subsections is accompanied by examples that give intuitions for the described structures.

    \subfile{grammar-formalisms/lcfrs.tex}
    \subfile{grammar-formalisms/dcp.tex}
    \subfile{grammar-formalisms/hybrid.tex}

    \ifSubfilesClassLoaded{%
        \printindex
        \bibliography{../references}%
    }{}
\end{document}
