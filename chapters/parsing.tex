\documentclass[../document.tex]{subfiles}


\begin{document}
    \chapter{Parsing via Supertagging}\label{chp:parsing}
    This chapter describes a unified framework of a supertagging-based parsing process with on the grammar formalisms explained in \cref{sec:grammars}.
    For this process, we assume an alphabet of tokens \(\varSigma\), an input sentence \(w = (w_i \in \varSigma \mid i \in [\ell])\) of length \(\ell \in \DN_+\) and a set of supertag blueprints \(B\).
    The parsing process is divided into four consecutive steps; the last step is an optional refinement procedure (reranking) as illustrated in \cref{fig:parsing:overview}:
    \begin{enumerate}
        \item\label{parsing:item:1}
            First, a sequence of supertag blueprints is predicted for each token in the input sentence; the number of predicted blueprints per position \(k \in \DN_+\) is a hyperparameter.
            This step requires a classifier that computes a score for each supertag blueprint at each position in the input sentence:
                We assume a parallel tagging model \(\varphi\colon \big(\bigcup_{i\in \DN_+} \varSigma^i \times B^i \big) \to \DR\) that computes the score \(\varphi^{(i)}(w, b) \in \DR\) for the input sentence \(w\), each supertag blueprint \(b \in B\) and position \(i \in [\ell]\).
            In our experiments, we use artificial neural networks to implement \(\varphi\); \cref{sec:preliminaries:nn} focuses on the concrete architectures used to realize the parallel tagging models.
            For the coming step \ref{parsing:item:2}, we consider two objects from this step:
            \begin{itemize}
                \item the family of predicted supertag blueprints \((b^{(i)}_{\kappa} \in B \mid i \in [\ell], \kappa \in [k])\) such that \((b^{(i)}_{\kappa} \in B \mid \kappa \in [k])\) is a sequence of \(k\) best supertag blueprints in \(\argmax^{(k)}_{b \in B} \varphi^{(i)}(w, b)\) for each position \(i\in [\ell]\), and
                \item the prediction confidence \((\varphi^{(i)}(w, b^{(i)}_{\kappa}) \mid i \in [\ell], \kappa \in [k])\) for each of these supertags.
            \end{itemize}
        \item\label{parsing:item:2}
            The predicted supertag blueprints are instantiated and combined into weighted grammars; the weight assignments are determined by the prediction confidence.
            A statistical parsing process for the underlying grammar formalism is used to find derivations for the sequence of consecutive positions in the input sentence.
            If reranking is used, then this step yields a sequence of \(n\) derivation trees, where \(n \in \DN_+\) is a hyperparameter and \emph{best} refers to an endorelation on the weights (e.g., highest score).
            Otherwise, it yields \emph{a best} derivation.
            We follow \citet{Auli12} and implement this step as an incremental process that considers supertags in growing confidence intervals \(\beta_1, \beta_2, \ldots\); that assumes an additional hyperparameters \((\beta_i \in \DR \mid i \in I)\) for some non-empty integer index set \(I\).
            This step is discussed in detail in \cref{sec:parsing}.
        \item\label{parsing:item:3}
            The derivations obtained in the previous step are converted into constituent structures by evaluating the \abrv{dcp} compositions or (in the case of \abrv{lcfrs} supertags) assuming the nonterminals as inner nodes.
            These constituent structures are extended to constituent trees by adding predicted \abrv{pos} symbols and the tokens from the input sentence.
        \item\label{parsing:item:4}
            Lastly, a reranking approach is used to (re-)score the constituent trees and pick the best as a result.
            We use \emph{discontinuous data-oriented parsing} \citep{CraSchBod16}, a precise statistical grammar formalism, to compute a (new) weight for each constituent tree obtained in the previous step; all previous weights are discarded during this step.
            The result is \emph{a best} constituent tree according to the new weight.
            This step is thouroughly investigated in \cref{sec:reranking}.
    \end{enumerate}

    \begin{figure}
        \subfile{figures/parsing-overview.tex}
        \caption{\label{fig:parsing:overview}
            Illustration of the supertagging-based parsing process.
            Staggered gray boxes indicate sequences of objects, e.g.\@, the \emph{parsing} step yields a sequence of \(n\) best derivation trees for the input sentence.
        }
    \end{figure}

    \subfile{parsing/dcp-parsing.tex}
    \subfile{parsing/reranking.tex}

    \ifSubfilesClassLoaded{%
        \printindex
        \bibliography{../refeferences.bib}%
    }{}
\end{document}
