\documentclass[../document.tex]{subfiles}

\begin{document}
    \chapter{Experiments}\label{sec:experiments}
    This chapter describes the experiments that we conducted for the extraction and parsing with supertags in three discontinuous constituent treebanks.
    First, \cref{sec:implementation} gives a short overview of the software that we used to conduct the experiments.
    \Cref{sec:hyperparameters} describes the options that are available to steer the supertag extraction and the training process for the classifiers used for supertagging.
    \Cref{sec:treebanks} glances on the three treebanks that we use to investigate the implementation.
    The meat of this chapter's content is an evaluation that shall find out which options for the supertag extraction work best in the treebanks.
    This search extends an earlier set of published experiments \citep{Rup22}, in particular with the newly added options for the extraction.
    It is described in detail for one of the three treebanks, \negra{}, in \cref{sec:gridsearch}.
    We also conduct a hyperparameter search for the other corpora with reduced sets of options, which are presented in \cref{sec:gridsearch:other}.
    We conclude the chapter with an overview of the parsing quality that we are able to achieve with the options found in the earlier sections.
    These results are compared to other state-of-the-art parsing technologies for discontinuous constituent trees.

    All experiments of our experiments were executed on a local compute server with an Intel Xeon Silver 4114 CPU (40 Cores each at 2.2 GHz), one NVidia GeForce RTX 2080 and two Tesla K80 GPUs.\footnote{
        The RTX 2080 is significantly faster than the K80 GPUs, but the latter have more memory which was needed to train the pretrained models with large transformer embeddings.
        We executed some sets of experiments in \cref{sec:gridsearch,sec:gridsearch:other} in parallel using both GPU models, so the reported time needed for the prediction might vary a bit.
        The results reported in \cref{sec:results} are all obtained using the RTX 2080. 
    }
    Our reported parse times are all sequential executions; the implementation is currently not parallelized but could easily be modified to, as the sentences are parsed independently of each other.
    
    \subfile{experiments/implementation.tex}
    \subfile{experiments/hyperparameters.tex}
    \subfile{experiments/treebanks.tex}
    \subfile{experiments/hyperparameter-search-in-negra.tex}
    \subfile{experiments/hyperparameter-search-in-tiger-and-dptb.tex}
    \subfile{experiments/results.tex}

    \ifSubfilesClassLoaded{%
        \printindex
        \bibliography{../references}%
    }{}
\end{document}