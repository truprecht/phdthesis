\documentclass[../document.tex]{subfiles}

\begin{document}
    \chapter{Experiments}\label{sec:experiments}
    This chapter describes the experiments we conducted for extracting and parsing with supertags in three discontinuous constituent treebanks.
    First, \cref{sec:implementation} gives a short overview of the software we used to conduct the experiments.
    \Cref{sec:models} explains the two neural network architectures that we use for the prediction of supertags and how they are trained.
    \Cref{sec:hyperparameters} describes the options available to steer the supertag extraction.
    \Cref{sec:treebanks} glances at the three treebanks we use to investigate the implementation.
    The meat of this chapter's content is an evaluation that shall determine which supertag extraction options work best for each of the three treebanks.
    This search extends an earlier set of published experiments \citep{Rup22}, particularly with the newly added options for the extraction procedure.
    It is described in detail for one of the three treebanks, \negra{}, in \cref{sec:gridsearch}.
    We also conduct a hyperparameter search for the other corpora with reduced options, presented in \cref{sec:gridsearch:other}.
    We conclude the chapter with an overview of the parsing quality that we can achieve with the options found in the earlier sections.
    These results are compared to other state-of-the-art parsing technologies for discontinuous constituent trees.

    All experiments of our experiments were executed on a local compute server with an Intel Xeon Silver 4114 CPU (40 Cores each at 2.2 GHz), one NVidia GeForce RTX 2080, and two Tesla K80 GPUs.\footnote{
        The RTX 2080 is significantly faster than the K80 GPUs, but the latter model has more memory, which was needed to train the pre-trained models with large transformer embeddings.
        We executed some sets of experiments in \cref{sec:gridsearch,sec:gridsearch:other} in parallel using both GPU models, so the reported time needed for the prediction might vary a bit.
        The results reported in \cref{sec:results} are all obtained using the RTX 2080.
    }
    Our reported parse times are all sequential executions; the implementation is not parallelized, but that could be easily changed, as the sentences are parsed independently.

    \subfile{experiments/implementation.tex}
    \subfile{experiments/prediction-models.tex}
    \subfile{experiments/hyperparameters.tex}
    \subfile{experiments/treebanks.tex}
    \subfile{experiments/hyperparameter-search-in-negra.tex}
    \subfile{experiments/hyperparameter-search-in-tiger-and-dptb.tex}
    \subfile{experiments/results.tex}

    \ifSubfilesClassLoaded{%
        \printindex
        \bibliography{../references}%
    }{}
\end{document}