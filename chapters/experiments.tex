\documentclass[../document.tex]{subfiles}

\begin{document}
    \chapter{Experiments}
    
    This chapter describes the experiments that we conducted for the extraction and parsing with supertags in three discontinuous constituent treebanks.
    First, \cref{sec:experiments:implementation} gives a short overview of the software that we used to conduct the experiments.
    \Cref{sec:experiments:hyperparameters} describes the options that are available to steer the supertag extraction and the training process for the classifiers used for supertagging.
    The meat of its content is an evaluation that shall find out which options for the supertag extraction work best in the treebanks.
    This search extends an earlier set of published experiments \citep{Rup22}, in particular with the newly added options for the extraction.
    It is described in detail for one of the three treebanks, NeGra, in \cref{sec:experiments:negra}.
    We also conduct a hyperparameter search for the other corpora with reduced sets of options, which are presented in \cref{sec:experiments:others}.
    We conclude the chapter with an overview of the parsing quality that we are able to achieve with the options found in the earlier sections.
    These results are compared to other state-of-the-art parsing technologies for discontinuous constituent trees.
    
    \subfile{experiments/implementation.tex}
    \subfile{experiments/hyperparameters.tex}
    \subfile{experiments/hyperparameter-search-in-negra.tex}
    \subfile{experiments/hyperparameter-search-in-tiger-and-dptb.tex}
    \subfile{experiments/results.tex}
\end{document}