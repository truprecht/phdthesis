\documentclass[../document.tex]{subfiles}

\begin{document}
    \chapter{Lexical Grammar Extraction from Treebanks}
    This chapter deals with the extraction of supertag grammars from treebanks.
    Here, the term \emph{supertag grammar} is understood as a lexical instance of any grammar formalism where the only terminal symbol occurring in the rules is a wildcard, which will be denoted by \tn{*}.
    Each rule in the instance of such a formalism is called a \emph{supertag}.
    In this chapter, we will consider the three previously discussed formalisms: \abrv{lcfrs}, \abrv{dcp}, and hybrid grammars.
    Beyond the previous definitions, there are two small extensions to \abrv{lcfrs} and \abrv{dcp}:
    \begin{compactenum}
        \item
            In the case of \abrv{lcfrs}, there is a boolean (i.e.\@ either it is present or not) label that accompanies each rule.
            It stores information from the lexicalization process and is needed to reverse the process in order to transform a derivation in the supertag grammar into a constituent tree.
        \item
            In the case of \abrv{dcp}, there is an integer value added to each rule.
            It will indicate the maximum fanout that is admissible during parsing.
    \end{compactenum}
    These extensions are considered an integral part of each supertag for the parsing process.
    Their purpose will be further discussed in \cref{sec:extraction:readoff,sec:extraction:guided}.
    These two additions will not be part in the case of instances of hybrid grammars, even though they are a combination of \abrv{lcfrs} and \abrv{dcp} grammars.

    There are two approaches for the extraction described in this chapter:
    \begin{compactenum}
        \item
            The process described in \cref{sec:extraction:readoff} first extracts a derivation in a simple \abrv{lcfrs} for each tree in a treebank.
            After that, the rules within each derivation are lexicalized by transporting the terminal symbols occurring in the leaves of the derivation into inner nodes.
            With the help of annotations that are added to the rules and nonterminals within the rules, this transportation is reversible.
        \item
            The extraction procedure described in \cref{sec:extraction:guided} first assigns a sentence position to each inner node of the constituent structure.
            In the following, it extracts a derivation of lexical grammar rules according to the positions assigned to the inner nodes in each subtree.
    \end{compactenum}

    \todo{Can the guided extraction be implemented for lexical lcfrs? -- the already existing marker `swapped` can be abused to give the transportation direction}
    \todo{describe some common topics: extracted grammar = set of all extracted rules, wildcard as terminal, sentence position as terminal during extraction}
    \todo{describe why binarization is a common topic for both extraction algorithms}
    \todo{discuss why we show both extractions if the first is an instance of the second: it is not that the guided extraction cannot be implemented for lcfrs rules + swapped annotation, but the different concept of transportation vs. guides}

    \subfile{extraction/binarization.tex}
    \subfile{extraction/lcfrs-lexicalization.tex}
    \subfile{extraction/guided-extraction.tex}

    \ifSubfilesClassLoaded{%
        \printindex
        \bibliography{../references}%
    }{}
\end{document}
