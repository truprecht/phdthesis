\documentclass[../document.tex]{subfiles}

\begin{document}
    \chapter{Lexical Grammar Extraction from Treebanks}\label{sec:extraction}
    This chapter deals with the extraction of supertag blueprints from treebanks.
    We will consider the following forms of supertag blueprints depending on the underlying grammar formalism:
    \begin{compactenum}
        \item
            Each \abrv{lcfrs} supertag blueprint is a tuple \((r, b)\) where \(r\) is a lexical \abrv{lcfrs} rule containing the terminal \wildcard{} and \(b \in \{\text{True}, \text{False}\}\) is a boolean marker.
            The marker \(b\) stores information from the lexicalization process and is needed to reverse the process to transform a derivation into a constituent tree.
        \item
            Each \abrv{dcp} supertag blueprint is a tuple \((r, f)\) where \(r\) is a lexical \abrv{dcp} rule containing the terminal \wildcard{} and \(f \in \DN_+\) is an integer.
            During the parsing process, \(f\) acts as a restriction for the fanout of all derivations rooted by a supertag instantiated for \((r,f)\).
        \item Each \abrv{hg} supertag blueprint is a lexical hybrid grammar rule containing the terminal \wildcard{}.
    \end{compactenum}
    For each of the three underlying grammar formalisms \(F\), an \deflab{supertag}[\(F\) supertag] is of the same form as an \(F\) supertag blueprint except the terminal \wildcard{} in the grammar rule is replaced by a positive integer.
    We call supertag that is obtained by replacing \wildcard{} in a supertag blueprint \(r\) with the integer \(i\) an \deflab{instance}[instance of \(r\)] and denote it by \(r[i]\).

    This chapter describes two approaches for the extraction of supertag blueprints from a constituent treebank:
    \begin{compactenum}
        \item
            The process described in \cref{sec:extraction:lcfrs} first extracts a derivation \abrv{lcfrs} rules for each tree in a treebank.
            After that, the rules within each derivation are lexicalized by transporting the terminal symbols occurring in the leaves into inner nodes.
            This transportation is reversible with the help of annotations that are added to the rules and nonterminals within the rules.
            This approach was the fruit of a collaboration with Richard Mörbitz \citep{MoeRup20,RupMoe21}.
        \item
            The extraction procedure described in \cref{sec:extraction:guided} first assigns a sentence position to each inner node of the constituent structure.
            The following extracts a derivation of lexical grammar rules according to the positions assigned to the inner nodes in each subtree.
            This approach extends the previous by parametrizing the lexical rule extraction but requires a more complex grammar formalism: hybrid grammars instead of \abrv{lcfrs}.
            The section builds upon our latest publication \citep{Rup22} and extends it by further formalizing the supertags and adding more options for the extraction parameters.
    \end{compactenum}
    These processes use the rank transformations described in \cref{sec:ranktransformations}.
    One of the transformations described in the section is binarization, a standard tool for parsing constituent trees.
    The other \emph{head-inward transformation} was not published before.


    \subfile{extraction/binarization.tex}
    \subfile{extraction/lcfrs-lexicalization.tex}
    \subfile{extraction/guided-extraction.tex}

%    \ifSubfilesClassLoaded{%
%        \printindex
%        \bibliography{../references}%
%    }{}
\end{document}
