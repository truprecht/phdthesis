\documentclass[../document.tex]{subfiles}

\begin{document}
    \chapter{Extraction of Lexical Grammars from Treebanks}\label{sec:extraction}
    This chapter deals with the extraction of supertag blueprints from treebanks.
    First, \cref{sec:supertags} investigates the form of supertags and supertag blueprints with the three grammar formalisms covered in \cref{sec:grammars}.
    The remainder of the chapter describes two approaches for the extraction of supertag blueprints from a constituent treebank:
    \begin{enumerate}
        \item
            The process described in \cref{sec:extraction:lcfrs} first extracts a derivation of \abrv{lcfrs} rules for each constituent tree in a treebank.
            After that, the rules within each derivation are lexicalized by transporting the terminal symbols occurring in the leaves into inner nodes.
            This transportation is reversible with the help of annotations that are added to the rules and nonterminals within the rules.
            This approach was the fruit of a collaboration with Richard Mörbitz \citep{MoeRup20,RupMoe21}.
        \item
            The extraction procedure described in \cref{sec:extraction:guided} first assigns to each inner node a sentence position of the constituent structure.
            It extracts a derivation of lexical grammar rules according to the positions assigned to the inner nodes in each subtree.
            This approach extends the previous by parametrizing the lexical rule extraction but requires a more complex grammar formalism: hybrid grammars instead of \abrv{lcfrs}.
            The section builds upon our latest publication \citep{Rup22} and extends it by further formalizing the supertags and adding more options for the extraction parameters.
    \end{enumerate}
    These processes use the rank transformations described in \cref{sec:ranktransformations}.
    One of the transformations described in the section is binarization, a standard tool for parsing constituent trees.
    The other \emph{head-inward transformation} was not published before.

    \subfile{extraction/1-supertags.tex}
    \subfile{extraction/2-binarization.tex}
    \subfile{extraction/3-lcfrs-lexicalization.tex}
    \subfile{extraction/4-guided-extraction.tex}

%    \ifSubfilesClassLoaded{%
%        \printindex
%        \bibliography{../references}%
%    }{}
\end{document}
