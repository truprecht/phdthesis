\documentclass[../../document.tex]{subfiles}

\begin{document}
    \section{Implementation}\label{sec:implementation}
    This section briefly overviews the software that implements the supertag extraction and parsing procedures described in the previous chapters.
    We developed the software over multiple years. Previous publications already used two iterations for parsing with \abrv{lcfrs} supertags \citep{RupMoe21,Rup22}.
    While the first of these iterations was developed in collaboration with my colleague Richard Mörbitz, all that came afterwards was my work.
    The implementation provides the following modules and functionalities:
    \begin{itemize}
        \item data structures for hybrid grammar and \abrv{dcp} supertags combined in a submodule called \texttt{grammar},
        \item procedures for the extraction of supertags from the common formats used to distribute discontinuous constituent treebanks, namely \negra{}'s \texttt{.export} format as well as \tiger{}'s \texttt{.xml} format, in a submodule called \texttt{grammar.extraction},
        \item a polymorphic (i.e., it can take on the shape of different architectures depending on the passed options) artificial neural network model used for a combined prediction of supertags and \abrv{pos} tags, as well as its training procedures in a submodule called \texttt{tagging},
        \item a unified parsing procedure for predicted supertags using the hybrid grammar and \abrv{dcp} formalisms in a submodule called \texttt{grammar.parsing},
        \item extraction and parsing procedures for the discontinuous data-oriented parsing approach; and functions for reranking predicted constituent trees in a submodule called \texttt{grammar.dop},
        \item a subcommand \texttt{grid} that runs every combination in a set of specified options for the extraction and training procedures.
    \end{itemize}

    Most data structures and extraction procedures are implemented in the Python programming language.
    We use the library provided by \texttt{disco-dop} \citep{Cra12} to read and preprocess treebanks, as well as the assessment of predicted constituent trees.
    \texttt{Pytorch} \citep{paszke2019pytorch} and \texttt{flair} \citep{Akb19} provide the implementations used for the construction and training of the \abrv{ann} model.
    The module for the parsing procedure is compiled using \texttt{cython} \citep{behnel2011cython} to enhance the parsing speed.
    In contrast to our earlier publications \citep{RupMoe21,Rup22}, the current version does not employ \texttt{disco-dop}'s parser for \abrv{lcfrs} anymore.
    We replaced it with a custom implementation, which specializes in parsing with lexical instances of hybrid grammars and \abrv{dcp}.
    The implementation is publicly available\footnote{\url{https://github.com/truprecht/disutapa}} as \emph{free software} licensed under GPLv2.
\end{document}