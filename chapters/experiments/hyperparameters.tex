\documentclass[../../document.tex]{subfiles}

\begin{document}
    \section{Extraction Hyperparameters}\label{sec:hyperparameters}
    This section describes the set of parameters that steer the procedures for the extraction of supertag blueprints.
    Some of them can be associated with the structure of the grammar rules contained in the extracted blueprints, such as the grammar formalism, rank transformation strategy and guide constructor, and others with the granularity of the extracted supertag blueprints, such as the Markovization parameters as well as the nonterminal constructors.
    The following list of gives a short description and viable values for each hyperparameter involved in extracting supertag blueprints.
    Generally, we will investigate configurations for all of these parameters by experimenting with each combination of parameter values.
    This is called grid search and is further explained in \cref{sec:gridsearch}.

    \paragraph*{Grammar Formalism.}
    Our experiments concern hybrid grammar and \abrv{dcp} supertags using the extraction procedure in \cref{sec:extraction:guided}.
    We omit \abrv{lcfrs} supertags as they are an instance of the hybrid grammar supertags that can be achieved with specific extraction parameters as described in \cref{sec:extraction:lcfrs}, namely the vanilla nonterminal constructor and the vanilla guide constructor.

    \paragraph*{Rank Transformation.}
    We consider rank transformations as an integral part of the extraction process for supertag blueprints from constituent treebanks.
    As discussed in \cref{sec:binarization}, they are specified by the following parameters:
    \begin{itemize}
        \item A transformation strategy determines the target ranks occurring in the trees after applying the rank transformation and the order/direction in which artificial nodes are  introduced. We distinguish four strategies: left-branching binarization, right-branching binarization, head-outward binarization, and head-inward transformation.
        \item Markovization contexts determine the information used to construct artificial nodes during the rank transformation process. This information is distinguished in horizontal and vertical Markovization context, denoted by \(h\) and \(v\), respectively. Both options are assigned to natural values \(h \ge 0\) and \(v \ge 1\). Usually, in grammar-based parsing, one would investigate combinations of values in the intervals \(0 \le h \le 3\) and \(1 \le v \le 2\). We saw in preliminary investigations that the number of extracted supertag blueprints rapidly rises with values beyond the minimum \(h = 0\) and \(v = 1\), negatively impacting the supertag prediction accuracy \citep{Rup22}. We will restrict our experiments to the following three combinations: \((h, v) \in \{(0,1), (1,1), (0,2)\}\).
        % \item Direction markers are an extension for the artificial nodes constructed during a rank transformation to illustrate the direction/order in which they were introduced. This extension has no effect when using a left- or right-branching binarization strategy, as markers would be fixed in those cases. That is not the case for the other two strategies, where such markers denote if a sub-derivation was extracted from a constituent to the left or to the right of a lexical head. In the case of the head-outward binarization strategy, this can be seen as a state that switches at a certain point during the derivation, where the bottom part is marked to be right-branching and the top part to be left-branching. This parameter's value is binary: whether the markers are present or not. We will investigate both values in tandem with the dependent and head guides.
        % \item A second extension for the rank transformations introduces a trailing unary node above the child that is processed last during the transformation at a node. E.g.\@ in right-branching binarization, this adds a trailing right-most binarization node. With all but the head guide constructor, this would be counter-productive as we expect constituent trees without unary nodes during the extraction of supertag blueprints. As in the previous case, this is a binary parameter, whether or not the trailing nodes are present. We will investigate both values in the experiments with the head guide constructor.
    \end{itemize}

    \paragraph*{Guide Constructors.}
    We investigate the six guide constructors introduced in \cref{sec:guides}: the vanilla, strict, near, least, dependent, and head guide constructors.
    While the first four define guides only using the given constituent structures, the latter two require a head assignment for each constituent tree in the treebank used for the extraction.
    We conduct experiments with all six options but separate them in \cref{sec:gridsearch:nts-guides,sec:gridsearch:head} to stress that the latter case requires additional information.

    \paragraph*{Nonterminal Constructors.}
    We conduct experiments using each of the three nonterminal constructors defined in \cref{sec:ntconstructors}, namely the vanilla, classic, and coarse nonterminal constructors.
    In the case of the coarse nonterminal constructor, we distinguish between the naive case (each nonterminal symbol is replaced by its first character) and the construction using a provided table.
    We will separate the experiments in the latter case into \cref{sec:gridsearch:coarse} to underline that such a table must be supplied as additional information to the constituent treebank.
    Moreover, in the case of \abrv{dcp} supertags, we investigate the nof and disc extensions for the nonterminal constructors that remove or change the fanout subscripts of constructred nonterminals, respectively.
\end{document}